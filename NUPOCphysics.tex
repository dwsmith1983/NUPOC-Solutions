%  Solutions to the physics section of NUPOC study guide
\section{Physics}

\begin{enumerate}
\item
  What must the angle \(\theta\) be in order for the block of mass \(M\) to
  start sliding when \(\mu = 0.8\)?
  \begin{figure}[H]
    \centering
    \subcaptionbox{What is giving}{
      \includestandalone[mode = image, width = 2in]{tikz/NUPOCphysics1}}
    \qquad
    \subcaptionbox{Forces drawn\label{solforce}}{
      \includestandalone[mode = image, width = 2in]{tikz/NUPOCphysics1sol}}
    \caption{The picture of the problem statement and forces drawn to help solve
      the problem.}
  \end{figure}
  Let \(\mathbf{N}\) be the normal force, \(\mathbf{F}_f = \mathbf{N}\mu\) be the
  frictional force, and \(M\mathbf{g}\) be the weight of the block.
  We will define our coordinate system such that positive x is in the direction
  the box will move once it over comes friction, and positive y in the direction
  of \(\mathbf{N}\).
  What we need to know now is \(\mathbf{F}_1\) and \(\mathbf{F}_2\).
  From the \cref{solforce}, we see that \(\mathbf{F}_1\) is parallel to line
  between \(M\mathbf{g}\) and \(\mathbf{F}_2\).
  From geometry, we can determine that the angle between \(M\mathbf{g}\) and
  \(\mathbf{F}_2\) is also \(\theta\).
  First, let's consider the \(90^{\circ}\) angle formed by \(\mathbf{F}_2\)
  and the ramp.
  From the red line to the ramp, we have a \(\theta\); therefore, the angle
  between the red line and \(\mathbf{F}_2\) is \(90 - \theta\).
  Additionally, we have that the angle between the red line and \(M\mathbf{g}\)
  is also \(90\).
  Let \(\psi\) be the angle between \(M\mathbf{g}\) and \(\mathbf{F}_2\).
  Then \(\psi + 90 - \theta = 90\).
  Thus, \(\psi = \theta\) as was needed to be shown.
  By determining the force from \(\mathbf{F}_2\) to \(M\mathbf{g}\), we have
  the force \(\mathbf{F}_1\) since \(\psi = \theta\).
  From the properties of a right triangle, we have that
  \begin{align*}
    \mathbf{F}_2 &= M\mathbf{g}\cos(\theta)\\
    \mathbf{F}_1 &= M\mathbf{g}\sin(\theta)
  \end{align*}
  By Newton's second law, \(\sum\mathbf{F} = m\mathbf{a}\), we have that
  \begin{subequations}
    \begin{align}
      Mg\sin(\theta) - N\mu &= M\ddot{x}\\
      N - Mg\cos(\theta) &= M\ddot{y}
    \end{align}
    \label{newton1}
  \end{subequations}
  Let's assume there is no acceleration.
  Then our \cref{newton1} can be written as
  \begin{subequations}
    \begin{align*}
      Mg\sin(\theta) &= N\mu\eqnumtag\label{xnoa}\\
      N &= Mg\cos(\theta)\eqnumtag\label{ynoa}\\
      \intertext{When we combine \cref{xnoa,ynoa}, we obtain the following
      expression.}
      \tan(\theta) &= \mu\\
      \tan(\theta) &= 0.8
    \end{align*}
  \end{subequations}
  Therefore, \(\theta > \arctan(0.8)\approx 38.6598^{\circ}\).
\item
  Find the final velocity of \(M\) for both elastic collision and inelastic
  collision.
  \begin{figure}[H]
    \centering
    \includestandalone[mode = image, width = 3in]{tikz/NUPOCphysics2}
    \caption{A ball of mass \(M\) with speed \(V_0\) collides with a ball of
      mass \(m\) and speed of \(v_0 = 0\).}
  \end{figure}
  For elastic collision, we have that the Conservation of Kinetic Energy and
  the Conservation of Momentum hold.
  \begin{alignat*}{2}
    \text{CoKE} &: \frac{1}{2}m_1v_1^2 + \frac{1}{2}m_2v_2^2 && ={}
    \frac{1}{2}m_1v_{1f}^2 + \frac{1}{2}m_2v_{2f}^2\\
    \text{CoM} &: m_1v_1 + m_2v_2 && ={} m_1v_{1f} + m_2v_{2f}
  \end{alignat*}
  Let's apply these definitions to our problem now.
  \begin{align*}
    MV_0^2 &= MV_{0f}^2 + mv_{0f}^2\\
    M(V_0^2 - V_{0f}^2) &= mv_{0f}^2\eqnumtag\label{coke2}\\
    \intertext{Recall that \(V_0^2 - V_{0f}^2 = (V_0 - V_{0f})(V_0 + V_{0f})\).}
    MV_0 &= MV_{0f} + mv_{0f}\\
    M(V_0 - V_{0f}) &= mv_{0f}\eqnumtag\label{com2}
  \end{align*}
  Let's divide \cref{coke2} by \cref{com2}.
  \[
  \frac{V_0^2 - V_{0f}^2}{V_0 - V_{0f}} = v_{0f}\Rightarrow
  V_0 + V_{0f} = v_{0f}
  \]
  Therefore, the velocity of \(M\) after collision is \(V_{0f} = v_{0f} - V_0\).
  For inelastic collision, the Conservation of Kinetic Energy doesn't hold.
  In an inelastic collision, the objects stick together after impact.
  Therefore, when we use the CoM equation, the mass of the final state will be
  \(m + M\).
  \[
  MV_0 = (M + m)V_{0f}\Rightarrow V_{0f} = \frac{MV_0}{M + m}
  \]
\item
  Describe the motion of the block-spring assembly when the block is displaced
  \(4\) inches from the equilibrium position.
  \begin{figure}[H]
    \centering
    \includestandalone[mode = image, height = 1.5in]{tikz/NUPOCphysics3}
    \caption{A block of mass \(M\) connected be a spring with spring constant
      \(k\).}
  \end{figure}
  In the x direction, we will have no movement; thus, the only direction of
  movement is in the y direction.
  In the y direction, according to Newton's second law, we have a simple harmonic
  oscillator.
  \[
  -ky = m\ddot{y}\Rightarrow y(t) = A\cos(\omega t) + B\sin(\omega t)
  \]
  where \(\omega = \sqrt{\frac{k}{m}}\).
  Prior to the block being released, the velocity is \(0\); that is,
  \(\dot{y}(0) = 0\).
  Since the block is displaced \(4\) inches, our initial condition for
  displacement is \(y(0) = -4\).
  Using the initial conditions, we have the following
  \begin{alignat*}{2}
    \dot{y}(0) &= B\omega && = {} 0\\
    y(0) &= A && = {} -4
  \end{alignat*}
  Therefore, the equation is \(y(t) = -4\cos(\omega t)\).
  When the block is displaced, it will oscillate with period
  \[
  \omega t = 2\pi\Rightarrow t = \frac{2\pi}{\omega} = 2\pi\sqrt{\frac{m}{k}}.
  \]
\item
  How does the gravitational force vary between two masses if distance is 
  doubled?
  How does electrostatic force vary between two charged particles if the
  distance is doubled?
  Explain using both equations and physical applications.
  \par\smallskip
  The gravitational force between two objects is
  \[
  F_g = \frac{Gm_1m_2}{r^2}.
  \]
  Let's determine what happens when we double the distance.
  Let \(r = 2r\).
  Then the denominator is \(4r^2\).
  By doubling the distance between the masses, the force decreases by \(4\).
  Since the electrostatic force between two charged particles is
  \[
  F_e = \frac{k\lvert q_1q_2\rvert}{d^2},
  \]
  doubling the distance will still decrease the force by \(4\).
\item
  Given the \cref{NUPOCphysics5}, calculate the distance traveled by the ball
  being thrown off the monument.
  \begin{figure}[H]
    \centering
    \includestandalone[mode = image, width = 1.5in]{tikz/NUPOCphysics5}
    \caption{A ball being thrown from a monument.}
    \label{NUPOCphysics5}
  \end{figure}
  For this problem, we will assume that there is no acceleration in the x
  direction, no air resistance, and \(v_{y0} = 0\).
  Let \((x_0, y_0) = (0, 0)\) and \(g = -9.8 \ m/s^2\). 
  \begin{align*}
    a_y(t) &= g\\
    \int a_y(t) \ dt &= \int g \ dt\\
    v_y(t) &= gt + v_{y0}\\
    \int v_y(t) \ dt &= \int(gt + v_{y0}) \ dt\\
    y(t) &= \frac{g}{2}t^2 + v_{y0}t + y_0
  \end{align*}
  By starting with acceleration in the y direction, we were able to integrate
  to obtain the position function for y.
  When \(y(t) = -555\) ft, we will have the time it takes for the ball to strike
  the ground.
  We need to convert feet to meters.
  The conversion for is \(\frac{1600 \ m}{5280 \ ft}\), so when
  \(y(t) = \frac{-1850}{111}\).
  \[
  y(t) = -4.9t_i^2 = \frac{-1850}{11}\Rightarrow t_i = 5.85857 \ s
  \]
  where \(t_i\) is the time of impact.
  Since we have no acceleration in the x direction, we have
  \begin{align*}
    a_x(t) &= 0\\
    v_x(t) &= v_{x0}\\
           &= 50\\
    x(t) &= 50t + x_0
  \end{align*}
  Let's convert \(50\) ft/sec to meters per sec.
  \(\frac{50 * 1600 \ m}{5280} = \frac{500}{33}\) m/s.
  \[
  x(t) = \frac{500}{33}t
  \]
  Finally, we can determine the distance the ball traveled from \(x(t_i)\).
  \[
  x(5.85857) = 88.7662 \ m
  \]
\item
  A spaceship is accelerating at \(1000 \ m/s^2\).
  How much force is required from the backthrusters to completely stop the
  spaceship?
  \par\smallskip
  Let \(M\) be the mass of the spaceship.
  By Newton's second law, we have that \(F_s = 1000M \ Kg\cdot m/s^2\).
  Therefore, the force required to stop the spaceship is
  \(1000M \ Kg\cdot m/s^2\).
  If spaceship was moving a some velocity, \(v_0\), it would take a force larger
  than \(1000M \ Kg\cdot m/s^2\) to stop the spacecraft.
\item
  Find \(h\) such that the car will make it around the loop without falling.
  Find the \(x\) that occurs when the car impacts the spring.
  \begin{figure}[H]
    \centering
    \subcaptionbox{Prior to being released\label{notreleased}}{
      \includestandalone[mode = image, height = .75in]{tikz/NUPOCphysics7}}
    \quad
    \subcaptionbox{At the apex of the loop.\label{apexloop}}{
      \includestandalone[mode = image, width = 1in]{tikz/NUPOCphysics7loop}}
    \caption{Cart moving through a loop from an inclined starting position.}
  \end{figure}
  When the cart is moving through the loop, we have centripetal acceleration;
  that is, \(a = \frac{v^2}{r} = \ddot{y}\).
  Now let's used Newton's second law.
  \begin{align*}
    M\ddot{y} &= N + Mg\\
    \frac{Mv^2}{r} &= N + Mg
  \end{align*}
  What is the minimum speed that the cart must travel to not fall off?
  The minimum speed occurs when \(N = 0\) so \(v^2 = gr\) which is also the
  minimum kinetic energy since
  \[
  \frac{1}{2}Mv^2 = \frac{1}{2}Mgr.
  \]
  We can now use the Conservation of Energy.
  The initial kinetic energy of the system is zero and the initial potential
  energy is \(Mgh\).
  \begin{align*}
    Mgh_i &= Mgh_f + \frac{1}{2}Mv_f^2\\
    Mgh &= 2Mgr + \frac{1}{2}Mgr\\
    h = \frac{5}{2}r
  \end{align*}
  The cart will make it around the loop when \(h\geq\frac{5}{2}r\).
  For the second part of the question, recall that the energy of a compressed
  spring is \(\frac{1}{2}kx^2\).
  Suppose the spring is located at \(h = 0\).
  Then by the Conservation of Energy, we have
  \[
  Mgh = \frac{1}{2}kx^2.
  \]
  Since \(h\geq\frac{5}{2}r\), we have that \(x\geq\sqrt{\frac{5Mgr}{k}}\).
\item
  What angle will give the maximum range for a projectile neglecting air
  resistance?
  What would happen if air resistance occurred?
  \begin{figure}[H]
    \centering
    \subcaptionbox{The flight path of a projectile.}{
      \includestandalone[mode = image, width = 2.5in]{tikz/NUPOCphysics8}}
    \qquad
    \subcaptionbox{Components of the velocity vector.}{
      \includestandalone[mode = image, height = 1in]{tikz/NUPOCphysics8v}}
    \caption{Projectile motion}
  \end{figure}
  We will assume the only acceleration is in the \(y\) direction and it is
  gravity, \(\mathbf{a} = -9.8 \ m/s^2\).
  \begin{align*}
    a_x &= 0\\
    v_x &= \int 0 \ dt\\
        &= v_{0x}\\
    x(t) &= \int v_{0x} \ dt\\
        &= v_{0x}t + x_0\eqnumtag\label{xeomp}\\
    a_y &= -9.8\\
    v_y &= -9.8\int dt\\
        &= -9.8t + v_{0y}\\
    y(t) &= -4.9t^2 + v_{0y}t + y_0\eqnumtag\label{yeomp}
  \end{align*}
  We now have our equations of motion, \cref{xeomp,yeomp}.
  Let \((x_0, y_0) = (0, 0)\).
  Then \cref{xeomp,yeomp} can be written as
  \begin{align*}
    x(t) &= v_{0x}t\\
         &= v\cos(\theta)t\\
    y(t) &= v\sin(\theta)t - 4.9t^2
  \end{align*}
  We need to determine the time it takes for \(y = 0\) or when the projectile
  impacts the ground.
  Distance is simply the rate times the time, \(d = r\cdot t\), where
  \(r = v\cos(\theta)\) and
  \[
  0 = v\sin(\theta) - 4.9t^2\Rightarrow t = \frac{v\sin(\theta)}{4.9}.
  \]
  Thus, distance is now a function of the angle, \(d(\theta) = rt\).
  To maximize the range, we need to find the maximum \(\theta\).
  Let's take the derivative of \(d\) and set it equal to zero.
  \begin{gather*}
    d' = \frac{v^2}{4.9}(\cos^2(\theta) - \sin^2(\theta)) = 0\\
    \tan(\theta) = 1
  \end{gather*}
  A projectile being fired only occurs when
  \(\theta\in\big[0, \frac{\pi}{2}\big]\) since we can always change our
  coordinates; therefore, \(\theta = \frac{\pi}{4}\).
  If we consider air resistance, an angle less than \(\frac{\pi}{4}\) will
  produce a maximum range.
  This is due to the fact that a lower trajectory reduces the time and distance
  over which the drag force due to air is acting.
  This is the basic understanding, but this isn't entirely true.
  An article, from 1997, by Dr.s Richard Price and Joseph Romano, has determined
  that this isn't always the case.
  The article can be found at
  \url{http://www.physics.rutgers.edu/~zapolsky/381/aim.pdf}.
  It goes on to say that drag is proportional the nth power of velocity,
  \(\text{drag}\propto v^n\).
  The critical value for \(n\) is \(n_{\text{crit}}\approx 3.5\).
  For \(n > n_{\text{crit}}\), angles greater than \(\frac{\pi}{4}\) can
  achieve a max range.
\item
  If a piece of paper is put on a full glass of water and inverted, what
  happens?
  Why?
  \par\smallskip
  By paper, I assume we mean something along the lines of an index card.
  The water will stay in the cup.
  This occurs because we have a higher air pressure pushing against the index
  card on the bottom of the cup.
  In the cup, there may be a small pocket of air, but it is of a lower air
  pressure.
  The force from atmospheric pressure holds the index card up, and the lower
  pressure in the glass prevents the waters weight from pushing the card down.
\item
  Given a hollow and a solid cylinder of equal masses that are placed on an
  inclined plane with both cylinders having equal radii.
  Which cylinder will reach the bottom of the plane first?
  \begin{figure}[H]
    \centering
    \includestandalone[mode = image, width = 2.5in]{tikz/NUPOCphysics10}
    \caption{A cylinder rolling down and inclined plane.}
  \end{figure}
  The kinetic energy of the cylinder is the energy of the translational motion
  plus the rotational energy.
  \[
  KE = \frac{1}{2}mv^2 + \frac{1}{2}I\omega^2
  \]
  where \(I\) is the moment of inertia and \(\omega\) is the angular velocity.
  Recall that the relation between angular and translational velocity is
  \(v = \omega\cdot R\).
  Since a rotation through an angle \(\theta\) causes the cylinder to travel,
  a distance is traveled which we refer to has arc length,
  \(s = \theta\cdot R\).
  Then \(\frac{ds}{dt} = v\) and \(\frac{d\theta}{dt} = \omega\).
  Let's use the Conservation of Energy.
  Initial we will have no kinetic energy; therefore, we have
  \[
  mgy(t) = \frac{1}{2}mv^2 + \frac{1}{2}\omega^2 =
  \frac{v^2}{2}\Big(m + I\frac{1}{R^2}\Big).
  \]
  By solving for velocity, we have
  \[
  v^2 = \frac{2gy(t)}{1 + I\frac{1}{mR^2}}.
  \]
  The vertical velocity is \(\frac{dy}{dt}\) where \(y = s\sin(\alpha)\), and
  the translational velocity is \(\frac{dy}{dt} = v\sin(\alpha)\).
  Then
  \[
  \frac{dy}{dt} = \sqrt{\frac{2g}{1 + I\frac{1}{mR^2}}}\sqrt{y(t)}\sin(\alpha)
  \]
  where \(y(0) = 0\).
  We know have a differential equation to solve.
  \[
  y(t) = \Bigg[\sqrt{\frac{2g}{1 + \frac{I}{mR^2}}}\frac{\sin(\alpha)t}{2}
  + C\Bigg]^2
  \]
  At \(y(0)\), \(C = 0\).
  \[
  y(t) = \frac{g\sin^2(\alpha)t^2}{2\big(1 + \frac{I}{mR^2}\big)}
  \]
  Suppose the height of the cylinder is \(h\).
  Then the time required to roll down the incline plane is
  \[
  t = \frac{\sqrt{2\big(1 + \frac{I}{mr^2}\big)}}{\sqrt{g}\sin(\alpha)}.
  \]
  Then the cylinder with the smallest \(\frac{I}{mR^2}\) reach the bottom
  first.
  Let's assume the the cylinder has constant density, the cylinder as length
  or height \(\ell\) if it is set on, and the cylinder has radius \(R\).
  Since the cylinder has constant density, \(\rho = 1\).
  We can calculate the moment inertia by
  \[
  I = \iiint_Dr^2 \ dm.
  \]
  Since \(\rho = \frac{m}{V}\), \(dm = \rho \ dV\).
  \begin{align*}
    I_s &= \int_0^h\int_0^{2\pi}\int_0^Rr^2r \ drd\theta dx\\
        &= \frac{R^4\pi h}{2}\\
    \intertext{We now need to know the mass of the solid cylinder.}
    m_s &= \int_0^h\int_0^{2\pi}\int_0^Rr \ drd\theta dx\\
        &= R^2\pi h
  \end{align*}
  Then
  \[
  \frac{I_s}{m_sr^2} = \frac{R^4\pi h}{2R^2\pi h R^2} = \frac{1}{2}.
  \]
  For the hollow cylinder, let the inner radius be \(a\) where \(0 < a < R\).
  \begin{align*}
    I_h &= \int_0^h\int_0^{2\pi}\int_a^Rr^2r \ drd\theta dx\\
        &= \frac{\pi h}{2}(R^4 - a^4)\\
    m_h &= \int_0^h\int_0^{2\pi}\int_0^Rr \ drd\theta dx\\
        &= \pi h(R^2 - a^2)
  \end{align*}
  Then
  \[
  \frac{I_h}{m_hr^2} = \frac{(R^4 - a^4)\pi h}{2R^2(R^2 - a^2)\pi h} = \frac{1}{2}\Big(1 + \Big(\frac{a}{r^2}\Big)^2\Big).
  \]
  Since \(a\neq 0\), \(\frac{I_h}{m_hr^2} > \frac{I_s}{m_sr^2}\).
  Therefore, the solid cylinder will reach the bottom first.
\item
  In \cref{NUPOCphysics11}, find the position of the electron when it hits the
  screen.
  Will it hit the screen?
  What two variables can you change to determine where the electron will hit?
  (Assume that \(d\) and \(L\) are fixed)
  \begin{figure}[H]
    \centering
    \includestandalone[mode = image, width = 2.5in]{tikz/NUPOCphysics11}
    \caption{An electron from a cathode ray tube TV moving towards the screen.}
    \label{NUPOCphysics11}
  \end{figure}
  Let's assume the electron has mass \(m\).
  Since the field \(E\) is downward and the electron is negative, there will be
  an upwards force on the electron.
  Since \(d = v_{0x}t\), we have that \(t = \frac{d}{v_{0x}}\) but
  \(t = \frac{L}{v_{0x}}\).
  The upward force on the electron is \(F_e = eE\).
  By Newton's second law,
  \[
  F_e = ma\Rightarrow a = \frac{eE}{m}.
  \]
  When the electron enters the field, \(v_{0y} = 0\) and \(y_0 = 0\).
  Then \(y(t) = a\frac{t^2}{2} = \frac{eEl^2}{2mv_{0x}^2}\).
  This is the distance above \(\frac{d}{2}\) where the electron strikes.
  If \(\frac{eEl^2}{2mv_{0x}^2} > \frac{d}{2}\), then
  \(\frac{eEl^2}{mv_{0x}^2} > d\) which is above the screen so the electron
  misses.
  Suppose \(E\) is uniform.
  Then \(E = \frac{+V}{d}\) and
  \[
  y(t) = \frac{e(+V)L^2}{2mdv_{0x}^2}.\eqnumtag\label{crt}
  \]
  In \cref{crt}, \(L\), \(m\), and \(d\) are fixed.
  By changing the top plate potential, \(+V\), and the velocity of the
  electron, we can alter the the location of the electron striking the screen.
\item
  Given the following data from a projectile, find the height of the parabola
  when \(t = 4\).
  \begin{figure}[H]
    \centering
    \includestandalone[mode = image, width = 2.5in]{tikz/NUPOCphysics12}
    \caption{The flight of the projectile in \(4\) seconds.}
  \end{figure}
  We will be assuming there is no air resistance.
  Let the initial position be \((0, 0)\) and the final position be
  \((300, 0)\).
  Acceleration due to gravity is \(g = -9.8 \ m/s^2\).
  \begin{alignat*}{2}
    y(t) &= y_0 + v_{0y}t + g\frac{t^2}{2}\\
    y(4) &= 4v_{0y} - 4.9\cdot 16 && = {} 0\\
    v_{0y} &= 19.6\\
    y'(t) &= 19.6 - 9.8t && = {} 0\\
    t &= 2
  \end{alignat*}
  When \(t = 2\), the projectile will reach its maximum height.
  \[
  y_{\max} = 19.6 \ m
  \]
  Thus, the height of the parabola is \(19.6 \ m\).
\item
  List and discuss Newton's Laws of Motion.
  \par\smallskip
  \textbf{Newton's first law:} When viewed in an inertial reference frame, an
  object either remains at rest or moves at a constant velocity, unless acted
  upon by an external force.
  \par\smallskip
  \textbf{Newton's second law:} \(\sum\mathbf{F} = m\mathbf{a}\)
  \par\smallskip
  \textbf{Newton's third law:} When one body exerts a force on a second body,
  the second body simultaneously exerts a force equal in magnitude and opposite
  in direction on the first body.
  \begin{figure}[H]
    \centering
    \includestandalone[mode = image, height = 2in]{tikz/NUPOCphysics13}
    \caption{Diagram of Newton's third law.}
  \end{figure}
  From Newton's third law, we have \(\mathbf{F}_{12} = -\mathbf{F}_{21}\).
\item
  A bullet with a mass of \(10 \ g\) and a velocity of \(1000 \ m/s\) embeds
  in a wooden block with a mass of \(1000 \ g\) suspended by a rope.
  How high will the block swing in the vertical direction.
  \begin{figure}[H]
    \centering
    \includestandalone[mode = image, height = 1.5in]{tikz/NUPOCphysics14}
    \caption{Ballistic pendulum}
  \end{figure}
  Since the bullet embeds in the block, we have inelastic collision.
  At \(t_0\), the bullet is moving at \(1000 \ m/s\) with a mass of \(10 \ g\),
  the block has \(v_{0p} = 0\), and vertical height \(h = 0\).
  Let the final height and velocity be \(h_f\) and \(v_{fp}\), respectively.
  We will assume there is no air resistance acting on the bullet.
  By Conservation of Momentum, we have \(p_{1i} + p_{2i} = p_f\).
  Let \(p_1 = m_bv_{0b}\) and \(p_2 = m_pv_{0p}\).
  Since the bullet embeds into the block, \(p_f = (m_b + m_p)v_f\).
  We can now solve for the final velocity.
  \[
  v_f = \frac{1000}{101} m/s
  \]
  For the Conservation of Energy, we will examine the system immediately after
  impact and when the pendulum reaches its maximum height.
  Immediately after impact, the pendulum has no potential energy.
  At maximum height, the pendulum has no kinetic energy.
  By the Conservation of Energy, we have
  \(KE_{ib} + PE_{ib} = KE_{fp} + PE_{fp}\).
  \[
  \frac{1}{2}(m_b + m_p)v_f^2 = (m_b + m_p)gh
  \]
  where \(g\) is gravity and \(h\) is the maximum height.
  The only unknown is \(h\); therefore, we can solve for \(h\).
  \[
  h = \frac{1}{2g}v_f^2 = \frac{1}{2\cdot 9.8}\Big(\frac{1000}{101}\Big)^2 m
  \]
\item
  Given the following rocket sled with initial velocity equal to \(v_0\), find
  the total distance the sled travels.
  \begin{figure}[H]
    \centering
    \includestandalone[mode = image, width = 2in]{tikz/NUPOCphysics15}
    \caption{Rocket sled}
  \end{figure}
  We will assume the sled has mass \(m\) and that there is no air resistance.
  At \(t = 0\), the sled is at rest at \(0, 0\).
  There will only be acceleration in the y direction, gravity.
  The velocity in the x and y direction are \(v_x = v_0\cos(\theta)\) and
  \(v_y = v_0\sin(\theta)\), respectively.
  We have integrated from acceleration and previously shown that
  \begin{align*}
    x(t) &= x_0 + v_xt + a_x\frac{t^2}{2}\\
    y(t) &= y_0 + v_yt + a_y\frac{t^2}{2}
  \end{align*}
  Using what we know, we have
  \begin{align*}
    x(t) &= v_0\cos(\theta)t\\
    y(t) &= v_0\sin(\theta)t - 4.9t^2
  \end{align*}
  When \(y = 0\), the sled will impact the ground.
  \[
  t_i = \frac{v_0\sin(\theta)}{4.9}s
  \]
  Therefore, the total distance the sled travels is
  \[
  x(t_i) = \frac{v_0^2\cos(\theta)\sin(\theta)}{4.9}m.
  \]
\item
  Find the time it takes to hit the ground.
  \begin{figure}[H]
    \centering
    \includestandalone[mode = image, height = 1.75in]{tikz/NUPOCphysics16}
    \caption{A ball with initial velocity \(v_{0x}\).}
  \end{figure}
  Again, we will assume we only have acceleration due to gravity in the y
  direction and there is no air resistance.
  Let the initial position of the ball be \((0, 0)\).
  The initial velocity in the y direction is \(v_{0y} = 0\).
  Thus, we have
  \begin{align*}
    x(t) &= v_{0x}t\\
    y(t) &= a_y\frac{t^2}{2}\\
         &= -4.9t^2
  \end{align*}
  The ball has to drop \(h \ m\) for it to strike the ground.
  \[
  t = \sqrt{\frac{h}{4.9}}s
  \]
\item
  A man has a velocity \(v_0 = 3 \ m/s\) and starts \(z \ m\) behind a bus with
  initial velocity \(v_{0b} = 0\) at \(t = 0\).
  The bus accelerates with acceleration \(a = 1 \ m/s^2\).
  Does the man catch the bus?
  \par\smallskip
  We will assume that the man is not accelerating and air resistance is
  negligible.
  Our equations of motion for the man and the bus are then
  \begin{align*}
    m(t) &= m_0 + v_0t + a_m\frac{t^2}{2}\\
         &= 3t\\
    b(t) &= b_0 + v_{0b} + a_b\frac{t^2}{2}\\
         &= z + \frac{t^2}{2}
  \end{align*}
  If the man is to catch the bus, the distance he travels must equal the
  distance the bus travels; that is, \(m(t) = b(t)\).
  \[
  \frac{t^2}{2} - 3t + z = 0
  \]
  By the quadratic equation, we have
  \[
  t = 3\pm\sqrt{9 - 2z}.
  \]
  So \(\frac{9}{2}\geq z\).
  In order for the man to catch the bus and barring the bus starting behind
  the man, \(0 < z \leq\frac{9}{2}\).
\item
  A block of mass \(M_1\) is attached by string to a support.
  The block is raised to a height of \(H\) and released.
  It then strikes a block of mass \(M_2\) on a frictionless surface.
  Find the velocity of the block \(M_2\), assuming a totally elastic collision.
  \begin{figure}[H]
    \centering
    \includestandalone[height = 1.75in, mode = image]{tikz/NUPOCphysics18}
    \caption{\(M_1\) positioned at a height of \(h\).}
  \end{figure}
  With elastic collision, we have the Conservation of Momentum and Kinetic
  Energy.
  The block of mass \(M_2\) has initial velocity of \(0\).
  Therefore, by the Conservation of Momentum and KE, we have
  \begin{align*}
    M_1v_{1i} + M_2v_{2i} &= M_1v_{1f} + M_2v_{2f}\\
    M_1(v_{1i} - v_{1f}) &= M_2v_{2f}\eqnumtag\label{momb}\\
    M_1v_{1i}^2 + M_2v_{2i}^2 &= M_1v_{1f}^2 + M_2v_{2f}^2\\
    M_1(v_{1i}^2 - v_{1f}^2) &= M_2v_{2f}^2\eqnumtag\label{keb}
  \end{align*}
  The difference of squares can be factor to
  \(v_{1i}^2 - v_{1f}^2 = (v_{1i} - v_{1f})(v_{1i} + v_{1f})\).
  Let's divide \cref{keb} by \cref{momb}.
  \[
  \frac{v_{1i}^2 - v_{1f}^2}{v_{1i} - v_{1f}} = v_{2f}
  \]
  So the final velocity of the second mass after collision is
  \(v_{2f} = v_{1i} + v_{1f}\).
\item
  Given the following setup, why will only one ball swing out?
  \begin{figure}[H]
    \centering
    \includestandalone[height = 1in, mode = image]{tikz/NUPOCphysics19}
    \caption{Newton's Cradle}
  \end{figure}
  First, assume each ball has mass \(m\).
  At rest, \(v_0 = 0\).
  Also, suppose \(n\) balls fly out instead of one.
  Then by the Conservation of Momentum \(mv = nmu\), and by the Conservation of
  Energy, \(\frac{1}{2}mv^2 = \frac{n}{2}mu^2\).
  We have two equations with two unknowns.
  \begin{align*}
    v &= nu\\
    v^2 &= nu^2
  \end{align*}
  It is easy to that the only solution is when \(u = v\) and \(n = 1\).
  Therefore, we have reached a contradiction and only one ball will fly out.
\item
  A \(10 \ g\) bullet with velocity of \(1000 \ m/s\) strikes a \(100 \ g\)
  block at rest.
  What is their combined velocity?
  Can you work the problem using the principle of Conservation of Momentum?
  Energy?
  \begin{figure}[H]
    \centering
    \includestandalone[height = 0.5in, mode = image]{tikz/NUPOCphysics20}
    \caption{A bullet embedding into a block.}
  \end{figure}
  We will assume the block is on a frictionless surface.
  This is a case of inelastic collision.
  Let \(m\) be the mass of the bullet and \(M\) the mass of the block.
  By Conservation of Momentum, we have
  \[
  mv + MV = (m + M)v_f.
  \]
  Since the block is initially at rest, our equation is
  \[
  10\cdot 1000 = 10\cdot 11v_f\Rightarrow v_f = \frac{1000}{11}m/s.
  \]
  Since we have inelastic, we can't use Conservation of Energy.
\item
  What is momentum and how is it related to Newton's second law?
  \par\smallskip
  Linear momentum is the product of mass times velocity of an object,
  \(\mathbf{p} = m\mathbf{v}\).
  Newton's second law is \(\sum\mathbf{F} = m\mathbf{a}\) but acceleration
  is the derivative of velocity.
  \[
  m\mathbf{a} = \frac{d}{dt}(m\mathbf{v}) = \frac{d}{dt}\mathbf{p} =
  \sum\mathbf{F}
  \]
\item
  What is the maximum altitude reached when \(W = 100 \ lb\) and
  \(v_0 = 100 \ ft/s\).
  \begin{figure}[H]
    \centering
    \includestandalone[height = 1in, mode = image]{tikz/NUPOCphysics22}
    \caption{An object moving up with velocity of \(v_0\).}
  \end{figure}
  Here we can use the Conservation of Energy.
  Initially the potential energy will be zero, and when the object reaches its
  maximum height, the kinetic energy will be zero.
  Thus, we have
  \[
  \frac{1}{2}mv^2 = mgh\Rightarrow h = \frac{v^2}{2g} = \frac{100^2}{64}ft.
  \]
\item
  A mass is dropped from a height of \(H\).
  What is the velocity of the mass just before it hits the ground?
  \par\smallskip
  For this problem, we can use the Conservation of Energy.
  Initially the kinetic energy is zero and the final potential energy is zero.
  \[
  mgh = \frac{1}{2}mv^2\Rightarrow v = \sqrt{2gh}
  \]
\item
  Consider the following pendulum system:
  \begin{figure}[H]
    \centering
    \includestandalone[height = 1.25in, mode = image]{tikz/NUPOCphysics24}
    \caption{Simple pendulum}
  \end{figure}
  \begin{enumerate}[label = (\alph*), ref = \theenumi{(\alph*)}]
  \item
    \label{24a}
    If the bob is release from rest, what is maximum velocity attained?
    \par\smallskip
    We will need to assume there is no air resistance and the bob has mass
    \(m\).
    Since the bob is released from rest, the initial potential energy is zero.
    At the bob's maximum height, the kinetic energy is also zero.
    By the Conservation of Energy, we have
    \[
    \frac{1}{2}mv^2 = mgh\Rightarrow v = \sqrt{2gh}.
    \]
  \item
    What assumptions are made in the answer?
    \par\smallskip
    The assumptions we made were that air resistance is negligible and the bob
    has mass \(m\).
  \item
    What difference does it make if the system is in vacuum?
    \par\smallskip
    By assuming we have no air resistance, we are assuming it is in a vacuum.
    Therefore, there would no difference.
  \item
    Suppose a second mass \(m\) was suspended at the lowest point, what would
    be the velocities of both masses after the collision?
    \par\smallskip
    Recall that for elastic collisions we can use the Conservation of Energy
    and Momentum.
    At \(t = 0\), the suspended bob has no potential energy, and the bob
    hanging at equilibrium has kinetic energy.
    Let the suspended bob have mass \(m_1\) and the bob hanging at
    equilibruim have mass \(m_2\).
    Then
    \begin{align*}
      m_1v_{1i} + m_2v_{2i} &= m_1v_{1f} + m_2v_{2f}\\
      m_1(v_{1i} - v_{2i}) &= m_2v_{2f}\eqnumtag\label{penmo}\\
      m_1v_{1i}^2 + m_2v_{2i}^2 &= m_1v_{1f}^2 + m_2v_{2f}^2\\
      m_1(v_{1i}^2 - v_{1f}^2) &= m_2v_{2f}^2\eqnumtag\label{penen}
    \end{align*}
    We can now divide \cref{penen} by \cref{penmo} and solve for the final
    velocities.
    \[
    v_{1i} + v_{1f} = v_{2f}
    \]
    From \cref{24a}, we have that the bobs initial velocity is
    \(v_{1i} = \sqrt{2gh}\).
    Thus, the final velocities are
    \begin{align*}
      v_{1f} &= v_{2f} - \sqrt{2gh}\\
      v_{2f} &= v_{1f} + \sqrt{2gh}
    \end{align*}
  \item
    What if the collision was non-elastic?
    \par\smallskip
    For inelastic collision, our Conservation equations become
    \begin{align*}
      m_1v_{1i} &= (m_1 + m_2)v_f\\
      m_1v_{1i}^2 &= (m_1 + m_2)v_f^2
    \end{align*}
    Then we have
    \[
    v_{1i} = v_f = \sqrt{2gh}.
    \]
  \end{enumerate}
\item
  Given a spring with displacement force \(F = e^x\), determine the energy
  required to move the block three units.
  \par\smallskip
  The force of the spring opposes \(F\) so the sum of forces is \(e^x - kx\).
  Then the work is
  \[
  W = \int_0^3(e^x - kx)dx = e^x - \frac{kx^2}{2}\big|_0^3
  = e^3 - \frac{9k}{2} - 1.
  \]
\item
  Define the following:
  \begin{enumerate}[label = (\alph*)]
  \item
    \textbf{Work:} a force is to do work when it acts on a body, and there is
    displacement of the point of application in the direction of the force.
  \item
    \textbf{Energy:} energy is a fundamental property of any physical object
    or system of objects which is used to describe and predict its interaction
    with other objects.
  \item
    \textbf{Power:} power is defined as the amount of energy consumed per unit
    time.
  \end{enumerate}
\end{enumerate}

%%% Local Variables:
%%% mode: latex
%%% TeX-master: t
%%% End:
