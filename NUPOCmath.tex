%  Solutions to the math section of NUPOC study guide
\section{Mathematics, Calculus, and Differential Equations}

\begin{enumerate}
\item
  What is a solution to the equation \((1 - y)^2 + 2xy = 0\).
  \par\smallskip
  It is apparent that \((1, 1)\) and \((1, 0)\) can't be solutions.
  Therefore, the only two solutions to consider are \((1, i)\) and \((i, 1)\).
  If we take \((i, 1)\), then our equation becomes
  \[
  2i = 0
  \]
  which isn't true.
  Thus, the solution must be \((1, i)\).
  Let's check to see.
  \begin{align*}
    (1 - i)^2 + 2i &= 0\\
    1 - 2i - 1 + 2i &= 0
  \end{align*}
  As we can see, everything checks out.
\item
  The set of all points \(P(x, y)\) in a plane, such that the difference of their
  distance from two fixed points is a positive constant is called?
  \par\smallskip
  Let's start be looking at a diagram of the situation.
  Also, note that we can always make a change of coordinates that puts our two
  fixed points on the x axis line.
  \begin{figure}[H]
    \centering
    \includestandalone[mode = image, width = 3in]{NUPOCmath2}
    \caption{The geometry of two fixed points from the point \((x, y)\).}
  \end{figure}
  Now, let \(P_1 = (-a, 0)\) and \(P_2 = (a, 0)\).
  We can find both \(d_1\) and \(d_2\) by use of the distance formula.
  \begin{gather*}
    d_1 = \sqrt{(x + a)^2 + y^2}\\
    d_1 = \sqrt{(x - a)^2 + y^2}
  \end{gather*}
  Using the premise of the question that the difference of their distance is a
  positive constant, we have
  \begin{align*}
    R &= \sqrt{(x + a)^2 + y^2} - \sqrt{(x - a)^2 + y^2}\\
    \intertext{Using algebra, we can begin to re-write our equation above.}
    \bigg(R + \sqrt{(x - a)^2 + y^2}\bigg)^2 &= (x + a)^2 + y^2\\
    R^2 + 2R\sqrt{(x - a)^2 + y^2} + (x - a)^2 + y^2 &= (x + a)^2 + y^2\\
    \frac{R}{2}\sqrt{(x - a)^2 + y^2} &= ax - \Big(\frac{R}{2}\Big)^2\\
    \Big(\frac{R}{2}\Big)^2\big((x - a)^2 + y^2\big)
      &= a^2x^2 - 2ax\Big(\frac{R}{2}\Big)^2 + \Big(\frac{R}{2}\Big)^4\\
    x^2\Bigg(\Big(\frac{R}{2}\Big)^2 - a^2\Bigg) + \Big(\frac{R}{2}\Big)^2y^2
      &= \Big(\frac{R}{2}\Big)^2\Bigg(\Big(\frac{R}{2}\Big)^2 - a^2\Bigg)\\
         \frac{x^2}{\big(\frac{R}{2}\big)^2} +
         \frac{y^2}{\big(\frac{R}{2}\big)^2 - a^2} &= 1\eqnumtag\label{ell}
  \end{align*}
  At this point, we have an equation for an ellipse.
  Unfortunately, we have one question that needs to be cleared up.
  Is \(\big(\frac{R}{2}\big)^2 - a^2 > 0\)?
  Given an arbitrary triangle with sides \(a\), \(b\), and \(c\), we know by the
  triangle inequality that the addition of two sides of the triangle is always
  greater than or equal to the remaining side.
  That is,
  \[
  a + b \geq c
  \]
  or any permutation of the sides.
  We can know apply this result to our triangle.
  \begin{align*}
    d_1 + d_2 &\geq 2a\\
    d_1 + 2a &\geq d_2\eqnumtag\label{t2}\\
    d_2 + 2a &\geq d_1\eqnumtag\label{t3}
  \end{align*}
  Let's re-write \cref{t2,t3}.
  \begin{align*}
    2a &\geq d_2 - d_1\\
    2a &\geq d_1 - d_2
  \end{align*}
  From these two equations, we have that \(2a\geq\lvert d_1 - d_2\rvert\).
  Since the distance between difference of the two fixed points is a positive
  constant, \(\lvert d_1 - d_2\rvert = R\).
  Thus,
  \[
  2a\geq R\Rightarrow a\geq \frac{R}{2}.
  \]
  Let's go back to \cref{ell}.
  Since we know now that \(a\geq \frac{R}{2}\), we must re-write \cref{ell}.
  \[
  \frac{x^2}{\big(\frac{R}{2}\big)^2} - \frac{y^2}{\big(\frac{R}{2}\big)^2 - a^2}
  = 1
  \]
  which is the equation of a hyperbola.
\item
  A propeller plane and a jet travel \(3000\) miles.
  The velocity of the plane is \(1/3\) the velocity of the jet.
  It takes the prop plane \(10\) hours longer to complete the trip.
  What is the velocity of the jet?
  \par\smallskip
  We can model the time it takes the prop plane to complete the journey as
  \(t_p = t_j + 10\), and the distance is simply the velocity times time.
  Since both the prop plane and the jet travels the same distance,
  \begin{align*}
    v_jt_j &= \frac{v_j}{3}(t_j + 10)\\
    t_j &= 5
  \end{align*}
  So we have found the time it take the jet to complete the trip is \(5\) hours.
  Since \(3000 = v_jt_j\) and we now know the time of the jet, we can find that
  the velocity is \(v_j = 600 \ mph\).
\item
  What is the center of \(x^2 + y^2 - 2x - 4y - 17 = 0\)?
  \par\smallskip
  For this problem, we will use the technique of completing the square.
  \begin{align*}
    x^2 - 2x + y^2 - 4y &= 17\\
    \intertext{For completing the square, we must add and subtract
    \(\big(\frac{b}{2}\big)^2\) to the quadratic polynomials.}
    x^2 - 2x + 1 - 1 + y^2 - 4y + 4 - 4 &= 17\\
    (x - 1)^2 + (y - 2)^2 &= 22
  \end{align*}
  We now have an equation of a circle centered at \((1, 2)\) with radius
  \(\sqrt{22}\).
\item
  Rewrite with nothing higher than order \(2\):
  \[
  \frac{a^4 + b^4}{a^2 + b^2}
  \]
  Without using Complex variables, neither the numerator nor the denominator
  readily factors.
  We could use long division, but there is another way.
  For instance, take the denominator \(a^2 + b^2\).
  What happens when we square it?
  Well, we get \(a^4 + b^4 + 2a^2b^2\).
  Therefore, \((a^2 + b^2)^2 - 2a^2b^2 = a^4 + b^4\).
  By making this substitution, we obtain
  \[
  a^2 + b^2 - \frac{2a^2b^2}{a^2 + b^2}
  \]
  which has no order higher than \(2\).
\item
  Simplify \(\frac{3 + 2i}{3 - 2i}\)
  \par\smallskip
  To simplify a complex fraction, we need to first multiple by the conjugate
  where the conjugate of \(a + bi\) is defined as \(a - bi\).
  \[
  \frac{(3 + 2i)^2}{(3 - 2i)(3 + 2i)} = \frac{5 + 12i}{11} 
  \]
\item
  Solve the following linear system
  \begin{align*}
    5x - 4y + 2z &= 0\\
    -3x + 4y &= 6\\
    x + 4z &= 6
  \end{align*}
  For this problem, we can use matrix row operations to determine the tuple
  \((x, y, z)\).
  \[
  \begin{bmatrix}
    5 & -4 & 2 & 0\\
    -3 & 4 & 0 & 6\\
    1 & 0 & 4 & 6
  \end{bmatrix}
  \]
  Our goal here is to obtain zeros under every pivot position where the pivot
  positions are \(a_{ii}\) where \(i = 1\), \(2\), and \(3\).
  Using row operations, let \(r_2 = r_2 + 3r_3\) where each row is written is
  lowest form.
  \[
  \begin{bmatrix}
    5 & -4 & 2 & 0\\
    0 & 1 & 3 & 6\\
    1 & 0 & 4 & 6
  \end{bmatrix}
  \]
  Then take \(r_1 = r_1 - 5r_3\).
  \[
  \begin{bmatrix}
    0 & 2 & 9 & 15\\
    0 & 1 & 3 & 6\\
    1 & 0 & 4 & 6
  \end{bmatrix}
  \]
  We can interchange row one and row three but let's wait on that.
  Finally, take \(r_1 = r_1 - 2r_2\).
  \[
  \begin{bmatrix}
    0 & 0 & 1 & 1\\
    0 & 1 & 3 & 6\\
    1 & 0 & 4 & 6
  \end{bmatrix}
  \]
  The top row tells use that \(z = 1\).
  From row two, we have that \(y = 3\), and from row three, \(x = 2\).
  That is, our tuple is \((2, 3, 1)\).
\item
  What is a logarithm?
  How is \(e\), the natural logarithm base, defined?
  \par\smallskip
  Let \(\log_a(b) = x\).
  Then the logarithm identifies the power \(a\) needs to be raised to equal
  \(b\).
  In other words, \(a^x = b\).
  The natural logarithm is log base, \(\log_e(a) = x\Rightarrow e^x = a\).
  In calculus, the natural log is defined as
  \[
  \ln(x) = \int_0^x\frac{1}{t} \ dt.
  \]
\item
  The number of square feet in a circle is equal to the number of in feet of
  the circle's circumference.
  What is the circle's radius?
  \par\smallskip
  By the premise, we have \(2\pi r = \pi r^2\).
  Solving for \(r\), we obtain \(r(r - 2) = 0\).
  So \(r = 0, 2\).
  Zero is a trivial solution so \(r = 2\).
\item
  \label{circ}
  Derive the equation of a circle around any point.
  \par\smallskip
  Let \((h, k)\) be the center of the circle and \((x, y)\) be a point on the
  circle.
  Let's define the radius to be \(r\).
  Then the distance between any point \((x, y)\) and the center is
  \[
  r = \sqrt{(x - h)^2 + (y - k)^2}\Rightarrow r^2 = (x - h)^2 + (y - k)^2
  \]
  which is the equation of a circle around the center \((h, k)\) with radius
  \(r\).
\item
  Given a closed box, where the length is twice the height, the width is \(10m\)
  less than the length, and the surface area is \(10\) times the width times the
  height, what are the dimensions?
  \par\smallskip
  The surface of a rectangular box is defined as
  \[
  SA = 2h\ell + 2w\ell + 2hw.
  \]
  For our box, \(\ell = 2h\), \(w = 2h - 10\), and \(SA = 10(2h - 10)h\).
  \begin{align*}
    SA &= 2h\ell + 2w\ell + 2hw\\
       &= 2\big(2h^2 + 4h^2 - 20h + 2h^2 - 10h\big)\\
       &= 16h^2 - 60h\\
    20h^2 - 100h &= 16h^2 - 60h\\
    4h^2 - 40h &= 0\\
    h(h - 10) &= 0
  \end{align*}
  Therefore, \(h = 10\) so \(\ell = 20\) and \(w = 10\).
\item
  Derive the quadratic equation.
  \par\smallskip
  The general form of a parabolic equation is \(ax^2 + bx + c = 0\).
  First, we will divide by \(a\) and then complete the square.
  \begin{align*}
    x^2 + \frac{b}{a}x + \frac{c}{a} &= 0\\
    \Big(x + \frac{b}{2a}\Big)^2 &= \frac{b^2}{4a^2} - \frac{c}{a}\\
    x + \frac{b}{2a} &= \pm\sqrt{\frac{b^2 - 4ac}{4a^2}}\\
    x &= \frac{-b\pm\sqrt{b^2 - 4ac}}{2a}
  \end{align*}
\item
  What geometric surface encloses the maximum volume with the minimum surface
  area?
  How do you prove it?
  \par\smallskip
  \textbf{This will be added in last.}
\item
  What type of smooth curve would go through these points: \((0, 4)\) and
  \((\pm 2, 0)\)?
  What would the equation be?
  \par\smallskip
  Since we are given the x-intercepts, we know the function is of the form
  \(f(x) = (x - 2)(x + 2) = x^2 - 4\) but this function has a y-intercept of
  \(-4\) not \(4\).
  If we take the negative of \(f(x)\), we obtain \(f(x) = -x^2 + 4\) which is a
  parabola that opens down with the appropriate x and y intercepts.
\item
  Find the area of \cref{15a,15b} using calculus and also derive the formulas for
  the volume of \cref{15c,15d}.
  \begin{enumerate}[label = (\alph*), ref = \theenumi{(\alph*)}]
  \item
    \label{15a}
    Triangle
    \par\smallskip
    For any triangle on the xy-plane, we can always perform a change in
    coordinates so that one of the vertices is at the origin and one edge is
    along the x axis.
    Therefore, we will only consider a triangle that his this geometry.
    \begin{figure}[H]
      \centering
      \includestandalone[mode = image, width = 3in]{NUPOCmath15a}
      \caption{An arbitrary triangle with an edge on the x axis starting at
        the origin.}
    \end{figure}
    Before we begin, let's perform our sanity check.
    The area of the triangle is \(A = \frac{x_2y_1}{2}\); therefore, the formula
    that we are going to derive from calculus better equal this.
    The integral defines the area under the curve.
    That is, we can come up with an integral over our domain that should be the
    area of a triangle.
    First, let's determine the lines \(OP1\) and \(P1P2\).
    \begin{gather*}
      OP1 \Rightarrow f_1(x) = \frac{y_1}{x_1}x\\
      P1P2 \Rightarrow f_2(x) = \frac{y_1}{x_1 - x_2}x -
      \frac{y_1x_2}{x_1 - x_2}
    \end{gather*}
    We can break up our integral over the domain of the triangle as
    \begin{align*}
      \int_0^{x_1}f_1(x)dx + \int_{x_1}^{x_2}f_2(x) \ dx
      &= \int_0^{x_1}\frac{y_1}{x_1}x \ dx + \int_{x_1}^{x_2}\bigg(
        \frac{y_1}{x_1 - x_2}x - \frac{y_1x_2}{x_1 - x_2}\bigg)dx\\
      &= \frac{y_1}{2x_1}x^2\bigg|_0^{x_1} + \frac{y_1}{x_1 - x_2}
         \bigg[\frac{x^2}{2} - x_2x\bigg|_{x_1}^{x_2}\\
      &= \frac{y_1x_1}{2} + \frac{y_1}{x_1 - x_2}\bigg(\frac{x_2^2 - x_1^2}{2}
         - x_2(x_2 - x_1)\bigg)\\
      &= \frac{y_1x_1}{2} + y_1x_2 - \frac{y_1(x_2 + x_1)}{2}\\
      &= \frac{y_1x_2}{2}
    \end{align*}
  \item
    \label{15b}
    Circle
    \par\smallskip
    From \cref{circ}, we previously determined the equation for a circle with
    center \((h, k)\) and radius \(a\) is
    \[
    a^2 = (x - h)^2 + (y - k)^2.
    \]
    Let's solve this equation for \(y\).
    Then \(y(x) = \pm\sqrt{a^2 - (x - h)^2} + k\).
    By performing a change of coordinates, we can write our equations as
    \(y(x) = \pm\sqrt{a^2 - x^2}\).
    That is, a circle with origin \((0, 0)\).
    Let's convert to polar coordinates.
    Let \(x = a\sin(\theta)\) where the bounds of integration go from
    \(x\in[-a, a]\) to \(\theta\in\big[\frac{-\pi}{2}, \frac{\pi}{2}\big]\).
    Moreover, we can use the symmetry property of the circle to take \(4\) times
    the integral with bound \(\theta\in\big[0, \frac{\pi}{2}\big]\).
    Then \(dx = a\cos(\theta) \ d\theta\).
    \begin{align*}
      4\int_0^{\pi/2}\sqrt{a^2 - a^2\sin^2(\theta)}(a\cos(theta) \ d\theta)
      &= 4a^2\int_0^{\pi/2}\sqrt{1 - \sin^2(\theta)}\cos(\theta) \ d\theta\\
      &= 4a^2\int_0^{\pi/2}\cos^2(\theta) \ d\theta\\
      \intertext{Recall that \(\cos^2(x) = \frac{1}{2}(1 + \cos(2x))\).
      Then}
      &= 2a^2\int_0^{\pi/2}(1 + \cos(2\theta)) \ d\theta\\
      &= \pi a^2 + \big[a^2\sin(2\theta)\big|_0^{\pi/2}\\
      &= \pi a^2
    \end{align*}
    Again, we can do a sanity check.
    The area of a circle is \(A = \pi r^2\).
    Since are radius is \(a\), we have obtained the correct formula for the area
    of a circle.
  \item
    \label{15c}
    Pyramid
    \par\smallskip
    Volume is the integral of the area.
    Let's take a look in the xz, yz, and xy plane of pyramid with length
    \(\ell\), width \(w\), and height \(h\).
    \begin{figure}[H]
      \centering
      \subcaptionbox{xy plane}{\includestandalone[mode = image, width = 2in]{NUPOCmath15cxy}}
      \subcaptionbox{xz plane}{\includestandalone[mode = image, width = 2in]{NUPOCmath15cxz}}
      \subcaptionbox{yz plane}{\includestandalone[mode = image, width = 2in]{NUPOCmath15cyz}}
      \caption{2D views of a 3D pyramid}
      \label{pyramid}
    \end{figure}
    The differential area \(\Delta z\) is simply \(A = x\cdot y\) where \(x\)
    and \(y\) depend on the height of pyramid.
    Let's find the line that goes through \(h\) and \(\frac{\ell}{2}\) and
    \(\frac{w}{2}\), respectively.
    \begin{gather}
      z(x) = -\frac{2h}{\ell}x + h\label{zx}\\
      z(y) = -\frac{2h}{w}y + h\label{zy}
    \end{gather}
    Since \(x\) and \(y\) depend on \(z\), we need to write \(x\) and \(y\) in
    terms of \(z\) from \cref{zx,zy}.
    Before we do, notice that \cref{zx,zy} only govern half of the pyramid in
    their respective plane.
    Therefore, \(A = (2x)(2y)\) to make up for this discrepancy.
    \begin{gather*}
      2x = \ell\bigg(1 - \frac{z}{h}\bigg)\\
      2y = w\bigg(1 - \frac{z}{h}\bigg)
    \end{gather*}
    So \(A = w\ell\big(1 - \frac{z}{h}\big)^2\).
    \begin{align*}
      V &= \int_DA \ dA\\
        &= w\ell\int_0^h\bigg(1 - \frac{z}{h}\bigg)^2dz\\
        &= w\ell\int_0^h \bigg(1 - 2\frac{z}{h} + \frac{z^2}{h^2}\bigg)dz\\
        &= w\ell\bigg(h - h + \frac{h}{3}\bigg)\\
        &= \frac{w\ell h}{3}
    \end{align*}
  \item
    \label{15d}
    Cone
    \par\smallskip
    In common usage, cones are assumed to be right circular.
    That is, the base is a circle of some radius \(r\) and right implies height
    of the cone runs through the center at right angles straight up.
    Then for a cone, the view from the xz and yx planes are the same.
    \begin{figure}[H]
      \centering
      \includestandalone[mode = image, width = 3in]{NUPOCmath15d}
      \caption{A right circular cone with height \(h\).}
    \end{figure}
    In this case, our differential area is circle that depends on its height;
    that is, the radius of the circle changes.
    Since the area of a circle is \(A = \pi r^2\) and our radius depends on
    \(z\), we have \(A(z) = \pi (r(z))^2\).
    Let's assume our circle has a radius of \(\ell\) at the base which is the top
    of the cone since it is inverted.
    Then we have a line from the origin to \(x, y = \pm\frac{\ell}{2}\).
    \begin{gather}
      z(x) = \frac{2h}{\ell}x\label{15dzx}\\
      z(y) = \frac{2h}{\ell}y
    \end{gather}
    Solving for \(z\) in \cref{15dzx}, we will obtain \(x = \frac{\ell}{2h}z\).
    As in the previous problem, this is only considering half of the cone.
    Therefore, \(r = 2x = \frac{\ell}{h}z\).
    \begin{align*}
      V &= \int_DhA \ dA\\
        &= \frac{\pi\ell^2}{h^2}\int_0^hz^2 \ dz\\
        &= \frac{\pi\ell^2h}{3}
    \end{align*}
  \end{enumerate}
\item
  Draw the following curves and find the area between them:
  \begin{enumerate}[label = (\alph*)]
  \item
    \(y = x^3\) and \(y = x^2\)
    \par\smallskip
    In order to find the area between the two curves, we need to determine the
    domain of interest.
    By setting the functions equal to each other, we will be able to find their
    points of intersection.
    In this problem, this is a trivial task and one can easily see that the
    domain is \(x\in[0, 1]\).
    \begin{figure}[H]
      \centering
      \includestandalone[mode = image, width = 3in]{NUPOCmath16a}
      \caption{}
    \end{figure}
    The area between to curves is
    \[
    \int_D(f(x) - g(x)) \ dx
    \]
    where \(f(x)\) is the top curve and \(g(x)\) is the bottom curve.
    \[
    \int_0^1(x^2 - x^3) \ dx = \frac{1}{3}x^3 - \frac{1}{4}x^4\Big|_0^1 =
    \frac{1}{12}
    \]
  \item
    \(y = x^2\) and \(y = x\)
    \par\smallskip
    As with the previous part, we have that \(x\in[0, 1]\).
    \begin{figure}[H]
      \centering
      \includestandalone[mode = image, width = 3in]{NUPOCmath16b}
      \caption{}
    \end{figure}
    In this case, we have
    \[
    \int_0^1(x - x^2) \ dx = \frac{1}{2}x^2 - \frac{1}{3}x^3\Big|_0^1 =
    \frac{1}{6}.
    \]
  \end{enumerate}
\item
  Plot \(f(x) = x^2 + x - 6\).
  Find the area between the x axis on the top, the line \(y = -4\) on the
  bottom, and the graph on each side.
  \par\smallskip
  When dealing with a parabola, we can try to factor, complete the square,
  and/or use the quadratic equation.
  In this problem, \(f(x)\) is factorable.
  \[
  f(x) = (x - 3)(x + 2)
  \]
  We now know that the x-intercepts are \(x = 3, -2\) and the y-intercept is
  \(y = -6\).
  To find the vertex of the parabola, we need to complete the square which leads
  to the following form of \(f(x)\).
  \[
  f(x) = \Big(x + \frac{1}{2}\Big)^2 - \frac{25}{4}
  \]
  Thus, the vertex is \(\big(-\frac{1}{2}, -\frac{25}{4}\big)\).
  \begin{figure}[H]
    \centering
    \includestandalone[mode = image, width = 3in]{NUPOCmath17}
    \caption{}
  \end{figure}
  Since we want to find the area of the both pieces of the divided parabola,
  let's start by first finding the total area of the parabola below the x
  axis.
  \[
  \int_{-3}^2(-x^2 - x + 6) \ dx =
  -\frac{1}{3}x^3 - \frac{1}{2}x^2 + 6x\Big|_{-3}^2 = \frac{125}{6}
  \]
  Now, once we find one of the other areas, we can subtract it from the total
  area to get the other side.
  We need to find where the function \(f(x) = -4\) intersects the parabola.
  \[
  x^2 + x - 6 = -4\Rightarrow x^2 + x - 2 = (x + 2)(x - 1)
  \]
  So intersection occurs at \(x = -2\) and \(x = 1\).
  \[
  A_{\text{below}} = \int_{-2}^1(-4 - x^2 - x + 6) \ dx =
  \int_{-2}^1(-x^2 - x + 2) \ dx = \frac{9}{2}
  \]
  Therefore, the area bounded by the x axis and \(f(x) = -4\) is
  \(A_{\text{above}} = \frac{49}{3}\).
\item
  Rotate \(y = \frac{1}{x}\) about the x axis and find the volume from \(1\)
  to infinity.
  \begin{figure}[H]
    \centering
    \includestandalone[mode = image, width = 3in]{NUPOCmath18}
    \caption{\(\frac{1}{x}\) rotated about the x axis on the domain
      \(x\in[0.5, 5]\).}
  \end{figure}
  In order to find the volume, we need to determine the differential we will
  use.
  Therefore, let's take a look at our solid in the xy plane.
  \begin{figure}[H]
    \centering
    \includestandalone[mode = image, width = 3in]{NUPOCmath18-2}
    \caption{A two dimensional view of the solid.}
  \end{figure}
  The volume of the solid is \(V = \int_D\text{A} \ dA\) where the area is
  \(\pi r^2\).
  As \(\Delta x\) moves from \([1, \infty)\), the differential circle changes
  accordingly.
  That is, \(r(x) = \frac{1}{x}\).
  \[
  V = \pi\int_1^{\infty}\frac{1}{x^2} \ dx = \frac{-\pi}{x}\bigg|_1^{\infty}
  = \pi
  \]
  This problem is traditionally known as Gabriel's Horn which has infinite
  surface area and finite volume.
\item
  Determine the area between two concentric circles of radii \(1\) and \(2\),
  respectively, using calculus.
  \par\smallskip
  Before we use calculus to find the area, we can determine the area from
  geometry.
  With this solution, we will be able to check are result from our calculus
  approach.
  The area between the two circles is simply the difference or
  \(A = \pi(r_1^2 - r_2^2) = 3\pi\).
  \begin{figure}[H]
    \centering
    \includestandalone[mode = image, width = 3in]{NUPOCmath19}
    \caption{The area between two concentric circles of radii \(1\) and \(2\).}
  \end{figure}
  The equations for two circles of radii \(1\) and \(2\), respectively, centered
  at the origin are
  \begin{align*}
    x^2 + y^2 & = 4\\
    x_1^2 + y_1^2 &= 1
  \end{align*}
  which we derived in \cref{circ}.
  Solving for \(y\), we have
  \begin{align*}
    y &= \pm\sqrt{4 - x^2}\\
    y_1 &= \pm\sqrt{1 - x_1^2}
  \end{align*}
  Due to the nature of the problem, polar coordinates are preferred.
  Therefore, let \(x = 2\sin(\theta)\) and \(x_1 = \sin(\theta)\).
  Then \(dx = 2\cos(\theta)d\theta\) and \(dx_1 = \cos(\theta)d\theta\).
  Additionally, our bounds of integration are now
  \(x\in[-2, 2]\to\theta\in\big[\frac{-\pi}{2}, \frac{\pi}{2}\big]\) which is
  only the right half of the circle.
  Since a circle is symmetric, we only need to integrate from
  \(\theta\in\big[0, \frac{\pi}{2}\big]\).
  This is only a quarter of the area so need to multiple this integral by \(4\).
  \[
  A = 4\int_0^{\pi/2}(4\cos^2(\theta) - \cos^2(\theta))d\theta =
  4\bigg[\pi - \frac{\pi}{4}\bigg] = 3\pi
  \]
\item
  Integrate the following:
  \begin{enumerate}[label = (\alph*), ref = \theenumi{(\alph*)}]
  \item
    \label{20a}
    \(\int(x\sin(x)) \ dx\)
    \par\smallskip
    To integrate the above integral, we need to use integration by parts.
    Recall that integration by parts is
    \[
    uv - \int v \ du.
    \]
    Let \(x = u\) and \(dv = \sin(x)\).
    Then \(dx = du\) and \(v = -\cos(x)\).
    \[
    -x\cos(x) + \int\cos(x) \ dx = -x\cos(x) + \sin(x) + C
    \]
  \item
    \(\int x\sqrt{x^2 - 4} \ dx\)
    \par\smallskip
    This is integral is done by u-substitution.
    Let \(u = x^2 - 4\).
    Then \(du = 2xdx\Rightarrow dx = \frac{du}{2}\).
    \[
    \frac{1}{2}\int\sqrt{u} \ du = \frac{1}{3}u^{3/2} + C =
    \frac{1}{3}(x^2 - 4)^{3/2} + C
    \]
  \item
    \(\int\frac{e^4 - \frac{3}{x}}{x^2} \ dx\)
    \par\smallskip
    Here we have a straight forward integral.
    All we need to do is write fraction as a difference of two fractions.
    \[
    \int\bigg(\frac{e^4}{x^2} - \frac{3}{x^3}\bigg)dx =
    \frac{-e^4}{x} + \frac{3}{2x^2} + C
    \]
  \item
    \(\int(e^{-x} + 3x^2) \ dx\)
    \par\smallskip
    Again, we have another trivial integral.
    \[
    \int(e^{-x} + 3x^2) \ dx = -e^{-x} + x^3 + C
    \]
  \item
    \(\int(x\sin^2(x) + x^3) \ dx\)
    \par\smallskip
    For this integral, recall \(\sin^2(x) = \frac{1}{2}(1 - \cos(2x))\).
    The
    \[
    \frac{1}{2}\int(x - x\cos(2x)) \ dx + \int x^3 \ dx =
    \frac{x^2}{4} - \frac{1}{2}\int x\cos(2x) \ dx + \frac{x^4}{4} + C
    \]
    As in the \cref{20a}, we need to use integration by parts again.
    Let \(u = x\) and \(dv = \cos(2x)\).
    Then \(du = dx\) and \(v = \frac{1}{2}\sin(2x)\).
    \[
    \frac{x^2 + x^4 - x\sin(2x) + C}{4} + \frac{1}{4}\int\sin(2x) \ dx =
    \frac{x^2 + x^4 - x\sin(2x) + C}{4} - \frac{\cos(2x)}{8}
    \]
  \item
    \(\int\sec(u)\tan(u) \ du\)
    \par\smallskip
    This is a standard substitution problem since
    \(\frac{d}{du}(\sec(u)) = \sec(u)\tan(u)\).
    \[
    \int\sec(u)\tan(u) \ du = \sec(u) + C
    \]
  \item
    \(\int xe^x \ dx\)
    \par\smallskip
    This is another integration by parts.
    Let \(u = x\) and \(dv = e^x\).
    Then \(du = dx\) and \(v = e^x\).
    \[
    xe^x - \int e^x \ dx = e^x(x - 1) + C
    \]
  \item
    \(\int(y + 3)(y + 1) \ dy\)
    \par\smallskip
    For this problem, it will serve us better to multiple the factored
    polynomial out.
    Thus, \(y^2 + 4y + 3 = (y + 3)(y + 1)\).
    \[
    \int(y^2 + 4y + 3) \ dy = \frac{y^3}{3} + 2y^2 + 3y + C
    \]
  \item
    \(\int_0^R\int_0^{\pi/2}\int_0^{\pi/2}r\sin(\theta) \ d\phi d\theta dr\)
    \par\smallskip
    Iterative integrals work exactly like single integrals.
    In these situations, we treat the additional parameter to what we are
    integrating over as a constant.
    \begin{align*}
      \int_0^R\int_0^{\pi/2}\int_0^{\pi/2}r\sin(\theta) \ d\phi d\theta dr
      &= \frac{\pi}{2}\int_0^R\int_0^{\pi/2}r\sin(\theta) \ d\theta dr\\
      &= -\frac{\pi}{2}\int_0^R\cos(\theta)\big|_0^{\pi/2} \ dr\\
      &= \frac{\pi}{2}\int_0^Rr \ dr\\
      &= \frac{\pi R^2}{4} + C
    \end{align*}
  \item
    \(\int(2x + 1) \ dx\)
    \par\smallskip
    This last integral is just a basic integration.
    \[
    \int(2x + 1) \ dx = x^2 + x + C
    \]
  \end{enumerate}
\item
  Take the derivative with respect to \(x\) of the following:
  \begin{enumerate}[label = (\alph*)]
  \item
    \(\cos^4(x)\sin(x)\)
    \par\smallskip
    A derivative of a product of functions requires the product rule,
    \[
    \frac{d}{dx}(f(x)g(x)) = f'g + fg'.
    \]
    Furthermore, since we have power of cosine, we will need to also use the
    chain rule,
    \[
    \frac{d}{dx}(f\circ g)(x) = f'(g(x))g'(x).
    \]
    Therefore, the derivative is
    \[
    \frac{d}{dx}\big(\cos^4(x)\sin(x)\big) =
    -4\cos^3(x)\sin^2(x) + \cos^5(x).
    \]
  \item
    \(\frac{ae^{-bx}}{cx^2}\)
    \par\smallskip
    For this problem, we can use the quotient or product rule.
    To use the product rule, we need to write the expression as
    \(\frac{a}{c}e^{-bx}x^{-2}\) and proceed as before, but I will use the
    quotient rule here,
    \[
    \frac{d}{dx}\bigg(\frac{f(x)}{g(x)}\bigg) = \frac{f'g - fg'}{g^2}.
    \]
    Thus, the derivative is
    \[
    \frac{d}{dx}\bigg[\frac{ae^{-bx}}{cx^2}\bigg] =
    \frac{-a(be^{-bx}x^2 + 2e^{-bx}x)}{cx^4} =
    \frac{-a(be^{-bx}x + 2e^{-bx})}{cx^3}.
    \]
  \item
    \(5x^4\)
    \par\smallskip
    Here the derivative is simply \(\frac{d}{dx}(5x^4) = 20x^3\).
  \item
    \(x\sqrt{x^2 - 4}\)
    \par\smallskip
    Again, we will use the product and chain rule.
    \[
    \frac{d}{dx}\big(x\sqrt{x^2 - 4}\big) = \sqrt{x^2 - 4} +
    \frac{x^2}{\sqrt{x^2 - 4}}
    \]
  \item
    \(\sin(x)\), \(\cos(x)\), \(\tan(x)\), \(\cot(x)\), \(\sec(x)\), and
    \(\csc(x)\)
    \begin{align*}
      \frac{d}{dx}(\sin(x)) &= \cos(x)\\
      \frac{d}{dx}(\cos(x)) &= -\sin(x)\\
      \frac{d}{dx}(\tan(x)) &= \sec^2(x)\\
      \frac{d}{dx}(\cot(x))
      &= \frac{d}{dx}\bigg(\frac{\cos(x)}{\sin(x)}\bigg)\\
                            &= \frac{-\sin^2(x) - \cos^2(x)}{\sin^2(x)}\\
                            &= -\csc^2(x)\\
      \frac{d}{dx}(\sec(x)) &= \sec(x)\tan(x)\\
      \frac{d}{dx}(\csc(x)) &= \frac{d}{dx}\bigg(\frac{1}{\sin(x)}\bigg)\\
                            &= \frac{-\cos(x)}{\sin^2(x)}\\
                            &= -\cot(x)\csc(x)
    \end{align*}
  \item
    \(\ln(x)\) and \(10^x\)
    \par\smallskip
    For the natural log, the derivative rule is
    \[
    \frac{d}{dx}(\ln(f(x))) = \frac{f'}{f}.
    \]
    Therefore, \(\frac{d}{dx}(\ln(x)) = \frac{1}{x}\).
    For the second expression, we can derive the rule.
    \begin{align*}
      y &= a^{f(x)}\\
      \ln(y) &= f(x)\ln(a)\\
      \frac{y'}{y} &= f'(x)\ln(a)\\
      y' &= yf'(x)\ln(a)\\
      y' &= a^{f(x)}f'(x)\ln(a)
    \end{align*}
    Thus, our second derivative is \(\frac{d}{dx}(10^x) = 10^x\ln(10)\).
  \item
    \(x + x^3 + \sin(x)\cos(x) + \sin(x)\)
    \par\smallskip
    This is should be a trivial exercise after doing the previous derivatives.
    \[
    \frac{d}{dx}f(x) = 1 + 3x^2 + \cos^2(x) - \sin^2(x) + \cos(x)
    \]
  \item
    \(x^5 + \cos(x)e^x + \sin\big(\frac{x^2}{3}\big)\)
    \par\smallskip
    Here the derivative is
    \[
    \frac{d}{dx}f(x) = 5x^4 + e^x\cos(x) - e^x\sin(x) +
    \frac{2x}{3}\cos\bigg(\frac{x^2}{3}\bigg).
    \]
  \item
    \(x^{1/2} + x^2\sin^2(x)\)
    \par\smallskip
    The final derivative is
    \[
    \frac{d}{dx}f(x) = \frac{1}{2\sqrt{x}} + 2x\sin^2(x) + 2x^2\sin(x)\cos(x).
    \]
  \end{enumerate}
\item
  What is an integral?
  How is it used?
  What is the difference between a definite and an indefinite integral?
  \par\smallskip
  An integral is an anti-derivative.
  For simplicity, one can think of it as the inverse of a derivative.
  Integrals are primarily used to find areas, volumes, and surfaces.
  A definite integral has bounds of integration which finds the constant of
  integration whereas indefinite integrals have no bounds.
  Since there are no bounds for the indefinite integrals, we must always
  remember to put \(+ C\) for the constant of integration.
\item
  What is a derivative?
  How is it used?
  What is a differential?
  What is the significance of the first and second derivatives?
  \par\smallskip
  By definition, a derivative is
  \[
  \lim_{\Delta x\to 0}\frac{f(x + \Delta x) - f(x)}{\Delta x}.
  \]
  In other words, the derivative is slope of the tangent line or we can say the
  instantaneous rate of change of the function.
  Derivatives can be used to find velocity if we differentiate position and
  acceleration when we differentiate velocity.
  A differential is an infinitesimal width of distance.
  The first derivative of a function identifies critical points that identify
  maximum and minimums.
  The second derivative of a function identifies critical points that identify
  inflection points which is where concavity changes.
\item
  Prove that the derivative of \(x^2\) is \(2x\).
  \par\smallskip
  To prove \(2x\) is the derivative of \(x^2\), we need to use the definition.
  \begin{align*}
    \lim_{\Delta x\to 0}\frac{(x + \Delta x)^2 - x^2}{\Delta x}
    &= \lim_{\Delta x\to 0}\frac{2(\Delta x)x + (\Delta x)^2}{\Delta x}\\
    &= \lim_{\Delta x\to 0}2x + \Delta x\\
    & = 2x
  \end{align*}
\item
  What is the \(\lim\limits_{x\to 0}\frac{\sin(x)}{x}\)?
  \par\smallskip
  When we take the limit as shown, we obtain \(\frac{0}{0}\) which is an
  indeterminant form.
  A technique we can use for indeterminate forms is L'H{\^o}pital's rule.
  \begin{align*}
    \lim_{x\to 0}\frac{\sin(x)}{x}
    &= \lim_{x\to 0}\frac{\frac{d}{dx}\sin(x)}{\frac{d}{dx}x}\\
    &= \lim_{x\to 0}\cos(x)\\
    &= 1
  \end{align*}
\item
  Be able to integrate or differentiate by using parts, chain rule, or quotient
  rule.
  \par\smallskip
  All techniques we used in previous problems.
\item
  Draw the following curves.
  Plot any max, min, and points of inflection.
  \begin{enumerate}[label = (\alph*)]
  \item
    \(f(x) = e^{-x^2}\)
    \par\smallskip
    To find a maximum or minimum, we need to set \(f'(x) = 0\).
    \[
    f'(x) = -2xe^{-x^2} = 0
    \]
    Since the exponential is never zero, we have \(-2x = 0\) or \(x = 0\).
    By testing points to the left and right of zero, we can determine what is
    occurring at \(x = 0\).
    \begin{align*}
      f'(1) &< 0\\
      f'(-1) &> 0
    \end{align*}
    Since \(f'(-1) > 0\), we know that the slope of the tangent line is positive
    to the right of zero, and since \(f'(1) < 0\), we know the slope is negative
    to the left of zero.
    That is, we have a slope that raises and then falls.
    Thus, we have a maximum at \(x = 0\).
    For inflection points, we need to set \(f''(x) = 0\).
    \[
    f''(x) = 2e^{-x^2}(2x^2 - 1)
    \]
    We then have \(2x^2 - 1 = 0\) so \(x = \pm\frac{1}{\sqrt{2}}\) as our
    inflection points.
    \begin{figure}[H]
      \centering
      \includestandalone[mode = image, height = 1.75in]{NUPOCmath27a}
      \caption{The plot of \(f(x) = e^{-x^2}\) from \(x\in[-3, 3]\).}
    \end{figure}
  \item
    \(f(x) = a\sin(x)\) for \(x\in[0, 2\pi]\)
    \par\smallskip
    Again, we need to find \(f'(x) = 0\) and \(f''(x) = 0\).
    \[
    f'(x) = a\cos(x) = 0
    \]
    So \(x = \frac{\pi}{2} + \pi k\) where \(k\in\mathbb{Z}\).
    Since we are only dealing \(x\in[0, 2\pi]\), \(k = 0\) and \(1\).
    therefore, \(x = \frac{\pi}{2}\) and \(\frac{3\pi}{2}\).
    The question remains now are these x points maxima or minima?
    Let's test a point to the left and right of both critical values to find out.
    \begin{align*}
      f'\big(\frac{\pi}{4}\big) &= +\\
      f'\big(\frac{3\pi}{4}\big) &= -\\
      f'\big(\frac{5\pi}{4}\big) &= +
    \end{align*}
    Therefore, at \(x = \frac{\pi}{2}\), we have a maximum, and at
    \(x = \frac{\pi}{3}\), we have a minimum.
    Next, we need to determine if we have any inflection points and if so
    where.
    \[
    f''(x) = -a\sin(x) = 0
    \]
    We have zeros when \(x = 0\), \(\pi\), and \(2\pi\).
    Since \(x = 0\) and \(2\pi\) are the endpoints, they can't be points of
    inflection.
    Thus, our only point of inflection occurs at \(x = \pi\).
    \begin{figure}[H]
      \centering
      \includestandalone[mode = image, height = 2.25in]{NUPOCmath27b}
      \caption{The plot of \(f(x) = a\sin(x)\) for different values of \(a\).}
    \end{figure}
  \item
    \(f(x) = e^{\pi/2}\)
    \par\smallskip
    Since the first and second derivatives are a constant, zero, we have no
    maxima, minima, or inflection points.
    \begin{figure}[H]
      \centering
      \includestandalone[mode = image, width = 3in]{NUPOCmath27c}
      \caption{The plot of \(f(x) = e^{\frac{\pi}{2}}\) which is a horizontal
        line.}
    \end{figure}
  \item
    \(f(x) = 3x^2 - 17x - 10\)
    \par\smallskip
    For the first derivative, we have
    \[
    f'(x) = 6x - 17 = 0.
    \]
    Our only critical point is then \(x = \frac{17}{6}\).
    \begin{align*}
      f'(0) &= -\\
      f'(3) &= +
    \end{align*}
    Therefore, \(x = \frac{17}{6}\) is a minimum.
    Since the second derivative is a constant, we have no inflection points.
    \begin{figure}[H]
      \centering
      \includestandalone[mode = image, width = 3in]{NUPOCmath27d}
      \caption{The plot of \(f(x) = 3x^2 - 17x -10\).}
    \end{figure}
  \item
    \(f(x) = x^3 - x^2\)
    \par\smallskip
    In this case, we have
    \begin{align*}
      f'(x) &= 3x^2 - 2x\\
      f''(x) &= 6x - 2
    \end{align*}
    The critical points for the first derivative are \(x = 0\) and
    \(\frac{2}{3}\).
    \begin{align*}
      f'(-1) &= +\\
      f'\big(\frac{1}{2}\big) &= -\\
      f'(1) &= +
    \end{align*}
    Therefore, \(x = 0\) is a maximum and \(x = \frac{2}{3}\) is a minimum.
    A change in concavity occurs at \(x = \frac{1}{3}\).
    \begin{figure}[H]
      \centering
      \includestandalone[mode = image, width = 3in]{NUPOCmath27e}
      \caption{The plot of \(f(x) = x^3 - x^2\).}
    \end{figure}
  \item
    \(f(x) = x^2e^{-x^2}\)
    \par\smallskip
    For this equation, our first and second derivatives are
    \begin{align*}
      f'(x) &= 2xe^{-x^2}(1 - x^2)\\
      f''(x) &= 2e^{-x^2}(2x^4 - 5x^2 + 1)
    \end{align*}
    Then the critical points are \(x = 0\) and \(\pm 1\).
    \begin{align*}
      f'(-2) &= +\\
      f'\big(\frac{-1}{2}\big) &= -\\
      f'\big(\frac{1}{2}\big) &= +\\
      f'(2) &= -
    \end{align*}
    We have maxima at \(x = \pm 1\) and a minimum at \(x = 0\).
    For our inflection points, we need to solve the quartic polynomial
    \[
    2x^4 - 5x^2 + 1 = 0.
    \]
    Let \(y = x^2\).
    Then \(2y^2 - 5y + 1 = 0\).
    By the quadratic equation,
    \[
    y = \frac{5\pm\sqrt{17}}{4}.
    \]
    Since \(y = x^2\), we have that our inflection points are at
    \[
    x = \pm\frac{\sqrt{5\pm\sqrt{17}}}{2}.
    \]
    \vspace*{-1cm}
    \begin{figure}[H]
      \centering
      \includestandalone[mode = image, width = 5in]{NUPOCmath27f}
      \caption{The plot of \(f(x) = x^2e^{-x^2}\).}
    \end{figure}
  \end{enumerate}
\item
  Analyze the curve \(y = 1 + e^{-x}\) by finding the first two derivatives,
  maxima, minima, and inflection points.
  \par\smallskip
  The first derivatives for \(y\) are
  \begin{align*}
    y' &= -e^{-x}\\
    y'' &= e^{-x}
  \end{align*}
  Since \(e^{\pm x} > 0\), we have no maximums (minimums) and inflection points.
  Additionally, as \(x\to\infty\), \(y\to 1\).
  Therefore, we have a horizontal asymptote at \(y = 1\).
  \begin{figure}[H]
    \centering
    \includestandalone[mode = image, height = 2in]{NUPOCmath28}
    \caption{The plot of \(y = 1 + e^{-x}\).}
  \end{figure}
\item
  Find the max or min of a parabola and determine if it is a max or min.
  \par\smallskip
  The most general parabola is \(f(x) = ax^2 + bx + c\).
  The first derivative is then
  \[
  f'(x) = 2ax + b.
  \]
  A maximum (minimum) will occur at \(x = \frac{-b}{2a}\).
  We have a few cases to consider in order to determine if we have a maximum
  (minimum).
  \begin{enumerate}[label = {Case \arabic*:}]
  \item
    Both \(a\), \(b > 0\).
    \par\smallskip
    In this case, \(x_c\) (x critical) is negative.
    \begin{align*}
      f'\big(\frac{-b}{a}\big) &= -\\
      f'(0) &= +
    \end{align*}
    Therefore, \(x = \frac{-b}{2a}\) is a minimum.
  \item
    \(a > 0\) and \(b < 0\).
    \par\smallskip
    In this case \(x_c\), is positive.
    \begin{align*}
      f'\big(\frac{-b}{a}\big) &= +\\
      f'(0) &= - 
    \end{align*}
    Therefore, \(x = \frac{-b}{2a}\) is a minimum.
  \item
    \(a < 0\) and \(b > 0\).
    \par\smallskip
    In this case \(x_c\), is positive.
    \begin{align*}
      f'\big(\frac{-b}{a}\big) &= -\\
      f'(0) &= + 
    \end{align*}
    Therefore, \(x = \frac{-b}{2a}\) is a maximum.
  \item
    Both \(a\), \(b < 0\).
    \par\smallskip
    In this case \(x_c\), is negative.
    \begin{align*}
      f'\big(\frac{b}{a}\big) &= +\\
      f'(0) &= - 
    \end{align*}
    Therefore, \(x = \frac{-b}{2a}\) is a maximum.
  \end{enumerate}
\item
  Using calculus, derive the formula for the exposed surface area of a ball
  floating in water.
  \par\smallskip
  First, let's consider the geometric formula of surface area for a sphere
  (ball).
  The surface area of a sphere is circumference times arc length,
  \[
  SA = 2\pi r\ell.
  \]
  For an arbitrary sphere, we need to determine what the radius and arc length
  are.
  Let' start with the arc length, \(\ell\).
  \begin{figure}[H]
    \centering
    \includestandalone[mode = image, width = 2in]{NUPOCmath30}
    \caption{An arc on interval \(x\in[a, b]\) with \(n = 4\).}
  \end{figure}
  Let the end points of the estimated arc be labelled \(x_i\) where
  \(i = 1\), \(2\), \(3\), \(4\), and \(5\).
  Then the distance of any line segment is
  \(d_k = \sqrt{(x_{i + 1} - x_i)^2 + (f(x_{i + 1}) - f(x_i))^2}\); therefore,
  the arc length, \(\ell\), is
  \[
  \sum_{k = 1}^Nd_k. \eqnumtag\label{arcl}
  \]
  By the Mean Value Theorem, we have that
  \[
  f'(c) = \frac{f(b) - f(a)}{b - a}
  \]
  on where \(f\) is continuous on the closed interval \([a, b]\),
  differentiable on the open interval \((a, b)\), and \(a < b\).
  Let \(a = x_i\) and \(b = x_{i + 1}\).
  Then
  \[
  f'(x_k) = \frac{f(x_{i + 1}) - f(x_i)}{x_{i + 1} - x_i}\Rightarrow
  f'(x_k)(x_{i + 1} - x_i) = f(x_{i + 1}) - f(x_i)
  \]
  Let \(\Delta x = x_{i + 1} - x_i\).
  Then we can write \cref{arcl} as
  \[
  \sum_{k = 1}^N\sqrt{(\Delta x)^2\big(1 + (f'(x_k))^2\big)} =
  \sum_{k = 1}^N\Delta x\sqrt{\big(1 + (f'(x_k))^2\big)}.
  \]
  As \(N\to\infty\), \(\Delta x\ll 1\) and we can write the summation as the
  integral
  \[
  \lim_{N\to\infty}\sum_{k = 1}^N\Delta x\sqrt{\big(1 + (f'(x_k))^2\big)} =
  \int_a^b\Delta x\sqrt{\big(1 + (f'(x))^2\big)} \ dx.
  \]
  Now, we have that the radius is
  \[
  r = \frac{f(x_i) + f(x_{i + 1})}{2},
  \]
  but since \(\Delta x\ll 1\), \(f(x_i)\approx f(x_{i + 1})\).
  Thus, \(r = f(x)\).
  Finally, we have that the total surface area of a sphere is
  \[
  SA = 2\pi\lim_{N\to\infty}f(x)\sum_{k = 1}^N\Delta x
  \sqrt{1 + \big(f'(x_k)\big)^2} 
  = 2\pi\int_a^bf(x)\sqrt{1 + \big(f'(x)\big)^2} \ dx.
  \]
  Let's find the total sphere area of an arbitrary sphere. 
  Suppose the sphere as radius, \(r = R\).
  Then in the xy plane the equation for the circle is \(x^2 + y^2 = R^2\) or
  \(y = f(x) = \pm\sqrt{R^2 - x^2}\).
  \begin{align*}
    f'(x) &= \frac{-x}{\sqrt{R^2 - x^2}}\\
    \big(f'(x)\big)^2 &= \frac{x^2}{R^2 - x^2}\\
    \sqrt{1 + \big(f'(x)\big)^2} &= \frac{R}{\sqrt{R^2 - x^2}}\\
    f(x)\sqrt{1 + \big(f'(x)\big)^2} &= R
  \end{align*}
  After grinding through the above algebra, we have a much easier integral to
  solve.
  \[
  SA = 2\pi R\int_{-R}^Rdx = 4\pi R^2
  \]
  Now the only question remains is how to find the total exposed surface area
  of a ball floating in water.
  For this, we will assume the ball is in placid water.
  That is, the amount of surface area exposed doesn't change.
  Let \(h\) be the height of the water line, and let \(y = 0\) correspond to
  the mid point on the ball.
  We have to two cases to consider, because when \(h = 0\), the surface area
  exposed is \(2\pi R^2\).
  \begin{enumerate}[label = {Case \arabic*:}]
  \item
    When \(h > 0\).
    \par\smallskip
    Since \(h\) corresponds to \(y\), we need to determine the x associated
    with this \(y\) for the bounds of integration.
    \[
    h = \pm\sqrt{R^2 - x^2}\Rightarrow x = \pm\sqrt{R^2 - h^2}
    \]
    Then the exposed surface area is
    \begin{align*}
      SA &= 2\pi R\int_{-\sqrt{R^2 - h^2}}^{\sqrt{R^2 - h^2}} \ dx\\
         &= 4\pi R\sqrt{R^2 - h^2}\eqnumtag\label{saexp}
    \end{align*}
  \item
    When \(h < 0\).
    \par\smallskip
    For this case, it will be \(2\pi R^2\) plus \cref{saexp}.
  \end{enumerate}
\item
  Solve the following differential equations:
  \begin{enumerate}[label = (\alph*)]
  \item
    \(y'' + 6y' + 9 = 5\)
    \par\smallskip
    To solve this differential equation, let's first combine the constants.
    \[
    y'' + 6y' + 4 = 0
    \]
    We can then treat this differential equation as quadratic polynomial.
    That is, \(m^2 + 6m + 4 = 0\).
    Then \(y(x) = Ae^{m_1x} + Be^{m_sx}\) so all we need to do now is find the
    zeros of the polynomial.
    \begin{gather*}
      m = -3\pm\sqrt{5}\\
      y(x) = Ae^{x(\sqrt{5} - 3)} + Be^{-x(3 + \sqrt{5})}
    \end{gather*}
  \item
    \(\frac{dN}{dt} = -2N\)
    \par\smallskip
    For this ODE, we can use separation of variables.
    \begin{align*}
      \int\frac{dN}{N} &= -2\int dt\\
      \ln(N) &= -2t + C\\
      N(t) &= Ce^{-2t}
    \end{align*}
  \item
    \(y' = xy^3\) where \(y(0) = 1\)
    \par\smallskip
    Again we can use separation of variables.
    \begin{align*}
      \int\frac{dy}{y^3} &= \int x \ dx\\
      \frac{-1}{2y^2} &= \frac{x^2}{2} + C\\
      y(x) &= \frac{1}{\sqrt{C - x^2}}
    \end{align*}
    Let's now use the initial condition to find the constant of integration.
    \[
    y(0) = \frac{1}{C} = 1
    \]
    Therefore, \(C = 1\) and the exact solution is
    \[
    y(x) = \frac{1}{\sqrt{1 - x^2}}.
    \]
  \end{enumerate}
\item
  For the following curve, plot the first and second derivatives.
  \begin{figure}[H]
    \centering
    \subcaptionbox{Problem statement plot}{
      \includestandalone[mode = image, width = 3in]{NUPOCmath32}}
    \quad
    \subcaptionbox{Solution plot}{
      \includestandalone[mode = image, height = 2.25in]{NUPOCmath32sol}}
    \caption{Plots for the problem and solution.}
  \end{figure}
\item
  Given \(80\) feet of fencing, what is the maximum area that you can enclose
  along a wall?
  \par\smallskip
  Here we are given an optimization problem.
  Let's construct a diagram of what we know.
  \begin{figure}[H]
    \centering
    \includestandalone[mode = image, width = 3in]{NUPOCmath33}
    \caption{A rectangular area enclosed along a wall.}
    \label{NUPOCmath33}
  \end{figure}
  We can use calculus or algebra to solve to solve this problem.
  \begin{enumerate}[label = {Case \arabic*:}]
  \item
    Algebra only.
    \par\smallskip
    The perimeter of our fenced in area is \(P = 2y + x = 80\) and the area
    is \(A = xy\).
    Therefore, we can write \(x = 80 - 2y\) and \(A = 80y - 2y^2\).
    \[
    2y(40 - y) = 0\Rightarrow y = 0, \ 40
    \]
    We have that y intercepts of the parabola are \(y = 0\) and \(40\).
    Since this parabola opens down, we have a maximum, and this maximum is
    when \(y\in(0, 40)\).
    Again, since we are dealing with a parabola, we know it is symmetric about
    line that bisects it at the vertex (maximum in our case).
    Therefore, \(y = 20\) and \(x = 40\) so the maximum area is
    \(A = 20\cdot 40 = 800\) square feet.
  \item
    Calculus.
    \par\smallskip
    With calculus, we will have the same equation for area, \(A = 80y - 2y^2\).
    \[
    \frac{dA}{dy} = 80 - 4y
    \]
    So the critical value is \(y = 20\) which is a maximum since the parabola
    opens down.
    Thus, \(x = 40\) and the maximum area is \(A = 800\) square feet.
  \end{enumerate}
\item
  Given the figure below, determine the value of \(x\) so when the corners are
  removed and flaps folded up, the five-sided box formed will have the maximum
  volume.
  \begin{figure}[H]
    \centering
    \includestandalone[mode = image, width = 3in]{NUPOCmath34}
    \caption{A rectanlge with length, \(L\) and width, \(W\).}
  \end{figure}
  What we have here is an optimization problem.
  Let \(\ell = L - 2x\) and \(w = W - 2x\).
  Then \(V = \ell w h\).
  \begin{align*}
    V(x) &= (L - 2x)(W - 2x)x\\
         &= 4x^3 - 2x^2(L + W) + LWx\eqnumtag\label{vbox}\\
    V' &= 12x^2 - 4x(L + W) + LW
  \end{align*}
  We need to find the x critical value the will maximize the volume.
  \[
  x = \frac{L + W\pm\sqrt{L^2 - LW + W^2}}{6}
  \]
  Let \(x_+ = \frac{L + W + \sqrt{L^2 - LW + W^2}}{6}\) and
  \(x_- = \frac{L + W - \sqrt{L^2 - LW + W^2}}{6}\).
  Since \cref{vbox} is a cubic with the leading power positive, \(4x^3\), we
  know from a sketch of cubics of this form that the maximum occurs when
  \(x\in(0, x_-)\) and a minimum is when \(x\in(x_-, x_+)\).
  Thus, the the volume of the box is maximized when \(x = x_-\).
\item
  Two runners start at a distance of \(10\) miles from each other.
  They run towards each other at a constant velocity of \(5\) mph.
  A fly takes off from runner one's nose at \(t = 0\).
  The fly has a constant velocity of \(20\) mph and flies between the runners.
  Find the total distance that the fly has traveled when the runners collide.
  \par\smallskip
  We will assume the runner's velocity behaves much like the Dirac delta
  function.
  That is, at the moment the time starts to roll from \(t = 0\) to \(t > 0\),
  the runners are moving at \(v_r = 5\) mph.
  Therefore, this problem has to potential solutions depending on your view.
  We will consider both cases.
  \begin{enumerate}[label = {Case \arabic*:}]
  \item
    At \(t = 0\), the runner is at \(5\) mph.
    \par\smallskip
    Since the runner is at \(5\) mph when the fly takes off, the fly is has a
    velocity of \(v_f = 25\) mph.
    The runners will collide in \(1\) hour since each runner will cover 5 miles
    in that time.
    Thus, the fly will have travelled \(25\) miles. 
  \item
    At \(t = 0\), the runner is at \(0\) mph.
    \par\smallskip
    In this scenario, the fly will only have a velocity of \(v_f = 20\) mph, so
    in \(1\) hour, the fly will have travelled \(20\) miles.
  \end{enumerate}
\item
  What is a Laplace transform, a Fourier transform, or a Taylor series?
  How are each used?
  \par\smallskip
  The Laplace transform is defined as
  \[
  \mathcal{L}\{f(t)\} = \int_0^{\infty}f(t)e^{-st} \ dt
  \]
  where \(s = \sigma + i\omega\).
  The Fourier transform is defined as
  \[
  \mathcal{F}\{f(x)\} = \int_{-\infty}^{\infty}f(x)e^{-2\pi ix\xi} \ d\xi.
  \]
  A Taylor series is when you represent a function as an infinite series around
  some point say, \(a\).
  For example, let \(f\) be an arbitrary continuous function with continuous
  derivatives; in other words, let \(f\) belong to \(C^{\infty}\).
  Then the Taylor series of \(f\) is
  \[
  \sum_{n = 1}^{\infty}\frac{f^{(n)}(a)}{n!}(x - a)^n
  \]
  where \(f^{(n)}\) is the nth derivative of \(f\).
  Laplace transforms are used to solve ordinary differential equations and to
  find transfer functions in control systems.
  The transfer function is defined as the Laplace transform of the output
  over the Laplace transform of the input with zero initial conditions.
  The Fourier transfer is used to solve partial differential equations on
  infinite domains.
  Taylor series are used in perturbations.
  We can expand the Taylor series of the function in question and take only
  the linear terms, quadratic, or etc. depending on the desired level of
  accuracy.
\item
  When do you use L'H{\^o}pital's rule?
  \par\smallskip
  L'H{\^o}pital's rule is used in solving limits that produces indeterminate
  forms.
  The indeterminate forms we will usually see are \(0/0\), \(\infty/\infty\),
  \(0\cdot\infty\), \(\infty - \infty\), \(0^0\), \(1^{\infty}\), and
  \(\infty^0\).
\item
  What is the probability of throwing one \(7\) with two dice?
  \par\smallskip
  The probability of rolling a \(7\) with two dice is simply the number of
  ways a \(7\) can be rolled divided by the number of ways two dice can be
  rolled.
  The number of ways to roll a seven is \(6\) and the total possible rolls is
  \(36\).
  Thus, the probability of rolling a \(7\) with two dice is \(\frac{1}{6}\).
\item
  If the population doubles in two years, how long does it take to triple?
  \par\smallskip
  We will assume that the change in population is proportional to a constant
  multiple of the population.
  \[
  \frac{dP}{dt} \propto kP\Rightarrow \int\frac{dP}{P} = k\int dt
  \]
  Then \(P(t) = Ce^{kt}\).
  We know that \(P(0) = P = C\) and \(P(2) = 2P\).
  \[
  P(t) = Pe^{kt}
  \]
  We now need to find \(k\) so we can determine the time that produces \(3P\).
  \begin{align*}
    2P &= Pe^{2k}\\
    2  &= e^{2k}\\
    k  &= \frac{1}{2}\ln(2)
  \end{align*}
  Let's find the appropriate time now.
  \begin{align*}
    3P      &= P\exp\Big(\frac{1}{2}\ln(2)t\Big)\\
    \ln(3)  &= \frac{1}{2}\ln(2)t\\
    t       &= 2\frac{ln(3)}{\ln(2)}\\
            &\approx 3.16993
  \end{align*}
  It will take approximately \(3.16993\) years for the population to triple.
\item
  Find \(f(x)\) which best describes the following graph where \(A\) represents
  area.
  \begin{figure}[H]
    \centering
    \includestandalone[mode = image, width = 3in]{NUPOCmath40}
    \caption{The plot of \(f(x)\).}
  \end{figure}
  Let \((x_1, y_1)\) be the point in the northeast corner of the rectangle.
  Since integral finds the area under the curve, we know that
  \[
  A = \int_0^{x_1}f(x) \ dx.
  \]
  Additionally, the area in the upper half is
  \[
  3A = \int_0^{x_1}(y_1 - f(x)) \ dx\Rightarrow 3A =
  x_1y_1 - \int_0^{x_1}f(x) \ dx.
  \]
  Therefore, \(A = \frac{1}{4}x_1y_1\) and
  \[
  \int_0^{x_1}f(x) \ dx = \frac{x_1y_1}{4}.
  \]
\item
  Use a first order differential equation to find the function to represent
  current with respect to time and to find the time constant to the circuit.
  \begin{figure}[H]
    \centering
    \includestandalone[mode = image, width = 3in]{NUPOCmath41}
    \caption{Electrical diagram}
  \end{figure}
  For \(t > 0\),
  \[
  V = iR + \frac{1}{C}\int i(t) \ dt\eqnumtag\label{rccirc}
  \]
  where \(i(t) = \frac{dq}{dt}\).
  We can differeniate \cref{rccirc} to obatin
  \[
  0 = \dot{i}R + \frac{i}{C}.
  \]
  We can solve for current as a function of time now since
  \(Rm + \frac{1}{C} = 0\) so \(i(t) = Ae^{\frac{-t}{RC}}\).
  When \(t = 0\), \(i(0) = A = i_0\).
  Thus, \(i(t) = i_0e^{\frac{-t}{RC}}\).
\item
  Show how to solve a differential equation with matrices.
  \par\smallskip
  Let's consider the matrix differential equation
  \(\dot{\mathbf{x}} = \mathbf{A}\mathbf{x}\) where \(\mathbf{A}\) is a square
  matrix.
  Let's assume that \(\mathbf{A}\) is diagonalizable.
  \[
  \mathbf{A} = \mathbf{P}\mathbf{D}\mathbf{P}^{-1}
  \]
  Then the solution to the matrix differential equation is
  \[
  \mathbf{x} = \sum_{i = 1}^Nc_ie^{\lambda_it}\mathbf{v_i}
  \]
  \(c_i\) are constants, \(\lambda_i\) is the ith eigenvalue of \(\mathbf{A}\),
  and \(\mathbf{v}_i\) is the ith eigenvector.
\item
  Find the sum of \(\sum\limits_{n = 1}^{100}n\).
  \par\smallskip
  Thanks to Gauss, we know that sums of this form can be found using the formula
  \[
  \sum\limits_{n = 1}^Nn = \frac{n}{2}(n + 1).
  \]
  Thus, \(\sum\limits_{n = 1}^{100}n = 5050\).
\item
  Given the figure below with uniform mass, what is the y coordinate of the
  center of gravity?
  \begin{figure}[H]
    \centering
    \includestandalone[mode = image, width = 3in]{NUPOCmath44}
    \caption{The height of both blocks are \(a\) and the indentation of the
      smaller block is \(a\).}
  \end{figure}
  Suppose the density is \(\rho\).
  Then \(m = V\rho\) where \(V\) is the volume.
  The volume of the bottom block is \(a\), and the volume of the top block is
  \(a(1 - a)\).
  Then \(m_b = a\rho\), \(m_t = a\rho(1 - a)\), and
  \(M = m_b + m_t = a\rho(2 - a)\).
  Now let's assume the x axis runs along the front and y along the side.
  Then the center for both the top and bottom blocks are \(x = \frac{1}{2}\).
  \[
  \bar{y} = \frac{M_x}{M} = \frac{1}{a\rho(2 - a)}
  \]
  For practice, let's find the x coordinate for the center of gravity.
  The center for the bottom block is \(y_b = \frac{1}{2}\) and the top is
  \(y_t = \frac{1 - a}{2}\).
  \[
  \bar{x} = \frac{M_y}{M} = \frac{1 + 1 - a}{2a\rho(2 - a)} =
  \frac{1}{2a\rho}
  \]
\item
  Describe how to classify differential equations.
  \par\smallskip
  Differential equations can be classified into three main
  categories--ordinary differential equations (ODE), partial differential
  equations (PDE), and differential algebraic equations (DAE).
  Additionally, ODE and PDE can be classified as linear, nonlinear, homogeneous,
  and nonhomogeneous.
  PDE can then be classified as elliptic, parabolic, and hyperbolic.
\item
  Solve \(\ddot{x} + 5\dot{x} + 6x = e^{-t}\).
  \par\smallskip
  First, let's determine the complementary solution; that is, the solution to
  the homogeneous differential equation.
  \[
  m^2 + 5m + 6 = 0\Rightarrow (m + 3)(m + 2) = 0
  \]
  Therefore, \(x_c = Ae^{-3x} + Be^{-2x}\).
  For the particular solution, let \(y_p = Ce^{-t}\).
  Then \(\dot{x}_p' = -Ce^{-t}\) and \(\ddot{x}_p' = Ce^{-t}\).
  \begin{align*}
    C(e^{-t} - 5e^{-t} + 6e^{-t}) &= e^{-t}\\
    2Ce^{-t} &= e^{-t}
  \end{align*}
  So \(C = \frac{1}{2}\) and \(x_p = \frac{1}{2}e^{-t}\).
  \[
  x(t) = Ae^{-3t} + Be^{-2t} + \frac{1}{2}e^{-t}
  \]
\item
  What is the Laplace transform of \(f(t) = t\)?
  \par\smallskip
  Let's apply the definition.
  \begin{align*}
    \mathcal{L}\{t\} &= \int_0^{\infty}te^{-st} \ dt\\
    \intertext{We will have to use integration by parts where \(u = t\),
    \(du = dt\), \(dv = e^{-st}\), and \(v = \frac{-1}{s}e^{-st}\).}
                     &= \frac{-t}{s}e^{-st}\Big|_0^{\infty} +
                        \frac{1}{s}\int_0^{\infty}e^{-st} \ dt\\
                     &= \frac{-1}{s^2}e^{-st}\Big|_0^{\infty}\\
                     &= \frac{1}{s^2}
  \end{align*}
\item
  Solve \(y'' + 4y' + 3y = \sin(x)\).
  \par\smallskip
  Again, we will start by finding \(y_c\).
  \[
  m^2 + 4m + 3 = 0\Rightarrow (m + 3)(m + 1) = 0
  \]
  So \(y_c = Ae^{-3x} + Be^{-x}\).
  Let \(y_p = C_1\sin(x) + C_2\cos(x)\).
  Then \(y_p' = C_1\cos(x) - C_2\sin(x)\) and
  \(y_p'' = -C_1\sin(x) - C_2\cos(x)\).
  \begin{align*}
    \sin(x)(3C_1 - 4C_2 - C_1) + cos(x)(3C_2 + 4C_1 - C_2) = \sin(x)\\
    \sin(x)(2C_1 - 4C_2) + cos(x)(2C_2 + 4C_1) = \sin(x)
  \end{align*}
  We can use linear algebra to find the coefficients \(C_1\) and \(C_2\).
  \[
  \begin{bmatrix}
    2 & -4 & 1\\
    4 & 2 & 0
  \end{bmatrix}\Rightarrow
  \begin{bmatrix}
    1 & -2 & \frac{1}{2}\\
    0 & 1 & \frac{-1}{5}
  \end{bmatrix}
  \]
  Therefore, \(C_2 = \frac{-1}{5}\) and \(C_1 = \frac{1}{10}\) so
  \(y_p = \frac{1}{10}\sin(x) - \frac{1}{5}\cos(x)\).
  \[
  y(t) = Ae^{-3x} + Be^{-x} + \frac{1}{10}\sin(x) - \frac{1}{5}\cos(x)
  \]
\item
  Solve \(\dot{x} = \frac{x}{k}\).
  \par\smallskip
  We can use separation of variables here.
  \[
  \int\frac{dx}{x} = \frac{1}{k}\int dt\Rightarrow \ln(x) = \frac{t}{k} + C
  \]
  Therefore, the solution is \(x(t) = Ce^{\frac{t}{k}}\).
\item
  Solve the general and specific homogeneous equation with derivatives:
  \[
  \frac{dy}{dx} + Ky = 10
  \]
  The homogeneous equation is \(\frac{dy}{dx} = -Ky\).
  We can use separation of variables here as well.
  \[
  \int\frac{dy}{y} = -K\int dt\Rightarrow y(t) = Ce^{-Kt}
  \]
  Let \(y_p = A\).
  Then \(y_p' = 0\).
  We have that \(KA = 10\Rightarrow A = \frac{10}{K}\).
  \[
  y(x) = \frac{10}{K} + Ce^{-Kt}
  \]
\item
  Explain how to solve the following differential equation \(A'' + A' + A = 0\).
  \par\smallskip
  The roots of the polynomial form the constants, \(\alpha\), to the general
  solution,
  \[
  A(x) = C_1e^{\alpha_1t} + C_2e^{\alpha_2t}.
  \]
\item
  Solve \(y - 3y' = 0\) with \(y(0) = 3\).
  \par\smallskip
  We have \(-3m + 1 = 0\) so \(y(x) = Ae^{\frac{1}{3}x}\).
  Using the initial condition, \(y(0) = A = 3\).
  \[
  y(x) = 3e^{\frac{1}{3}x}
  \]
\end{enumerate}

%%% Local Variables:
%%% mode: latex
%%% TeX-master: t
%%% End:
