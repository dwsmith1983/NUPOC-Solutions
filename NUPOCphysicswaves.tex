%  Solutions to the wave section of NUPOC study guide
\subsection{Wave Properties and Oscillations}

\begin{enumerate}
\item
  What is the oscillation period of a simple harmonic oscillator?
  \par\smallskip
  A simple harmonic oscillator can be defined as a mass on a spring.
  Then by Hooke's law and Newton's second law, we have
  \[
  m\ddot{x} = -kx.
  \]
  The solution to the following ODE is
  \(x(t) = A\cos(\omega t) + B\sin(\omega t)\) where
  \(\omega = \sqrt{\frac{k}{m}}\).
  Let \(t\) be the period of the oscillations.
  Then
  \[
  \omega t = 2\pi\Rightarrow t = \frac{2\pi}{\omega}
  \]
  so \(t = 2\pi\sqrt{\frac{m}{k}}\).
\item
  Derive the period of a simple pendulum?
  \begin{figure}[H]
    \centering
    \includestandalone[height = 1.25in, mode = image]{NUPOCphysicswaves2}
    \caption{A simple pendulum with a small angle approximation.}
    \label{NUPOCphysicswaves2}
  \end{figure}
  For this problem, we will assume there is no air resistance and a small angle
  approximation which will become apparent later.
  From the \cref{NUPOCphysicswaves2}, we see that \(x = \ell\sin(\theta)\) and
  \(y = \ell(1 - \cos(\theta))\).
  We can use Newton's second law for both the x and y motion.
  \begin{align*}
    m\ddot{x} &= F_x\\
              &= -T\sin(\theta)\\
    m\ddot{y} &= F_y\\
              &= T\cos(\theta) - mg
  \end{align*}
  Now we need to find \(\ddot{x}\) and \(\ddot{y}\).
  \begin{align*}
    \dot{x} &= \ell\dot{\theta}\cos(\theta)\\
    \ddot{x} &= \ell\ddot{\theta}\cos(\theta) -
                \ell\dot{\theta}^2\sin(\theta)\\
    \dot{y} &= \ell\dot{\theta}\sin(\theta)\\
    \ddot{y} &= \ell\ddot{\theta}\sin(\theta) + \ell\dot{\theta}^2\cos(\theta)
  \end{align*}
  We can now construct our equations of motion.
  \begin{align}
    m\ell(\ddot{\theta}\cos(\theta) - \dot{\theta}^2\sin(\theta))
    &= -T\sin(\theta)\label{simpenx}\\
    m\ell(\ddot{\theta}\sin(\theta) + \dot{\theta}^2\cos(\theta))
    &= T\cos(\theta) - mg\label{simpeny}\\
    \intertext{We can now multiple \cref{simpenx} by cosin, \cref{simpeny}
    by sine, and then add \cref{simpenx,simpeny} together.}
    \ddot{\theta} + \frac{g}{\ell}\sin(\theta) &= 0\label{simpen}
  \end{align}
  \Cref{simpen} is a nonlinear equation.
  Since we are assuming small angle approximations, \(\theta\ll 1\), let's
  look at the power series for sine.
  \[
  \sin(x) = \sum_{n = 1}^{\infty}\frac{(-1)^{n - 1}x^{2n - 1}}{(2n - 1)!}
  = x - \frac{x^3}{3!} + \frac{x^5}{5!} - \cdots
  \]
  Therefore, \(\sin(\theta)\approx\theta\) and \cref{simpen} is
  \[
  \ddot{\theta} + \frac{g}{\ell}\theta = 0.
  \]
  The solution to ordinary differential equation is
  \(\Theta(t) = A\cos(\omega t) + B\sin(\omega t)\) where
  \(\omega = \sqrt{\frac{g}{\ell}}\).
  Then the period is \(\omega t = 2\pi\) so \(t = 2\pi\sqrt{\frac{\ell}{g}}\).
\item
  Explain the difference between light and radio waves.
  \par\smallskip
  The main difference is waves length.
  Radio waves start around \(1 \ cm\) whereas visible light is between
  \(380 \ nm\) and \(740 \ nm\).
  Additionally, radio waves are created by the acceleration of electrons in
  radio antenna, and light waves are created by the oscillation of the
  electrons within the atoms.
\item
  \label{w4}
  What is the relationship between frequency and wavelenght?
  \par\smallskip
  Frequency is \(f = \frac{1}{t}\) where \(t\) is the period, and the wave
  length \(\lambda\) is defined as \(\lambda = \frac{c}{f}\) and \(c\) is the
  speed of light.
\item
  What is the frequency of a \(5\) \AA{} wavelength emission?
  \par\smallskip
  Since \(5\) \AA{} is a wavelength, we know from \cref{w4} that
  \(5 \AA = \frac{c}{f}\).
  \[
  f = \frac{c}{5\text{\AA}} = \frac{3\times 10^8 \ m/s}{5\times 10^{-10} \ m}
  = \frac{3}{500s}
  \]
  Therefore, the frequency is \(\frac{3}{500} \ Hz\).
\item
  Contrast light and sound waves.
  How do they propagate energy?
  Do they travel at different speeds in different media?
  Why?
  \par\smallskip
  Sound requires a meduim to propage.
  Progation occurs from vibration which produces a mechanical wave of pressure
  and displacement through the medium.
  Light doesn't require a medium.
  Propation occurs from the vibration of an electric charge.
  Additionally, light travels exremely fast, \(c = 3\times 10^8 \ m/s\).
  Light waves are electromagnetic waves consisting of varying electric and
  magnetic fields.
  Yes, when waves (sound or light) enter a medium, they may slow down or speed
  up.
\item
  Define Doppler Shift.
  \par\smallskip
  The Doppler Shift is a change in frequency due to the Doppler Effect.
  Therefore, we need to understand what the Doppler Effect is.
  The Doppler Effect is an increase (or decrease) in the frequency of sound,
  light, or other waves as the source and the observer move toward (or away
  from) each other.
  The effect causes the sudden change in pitch noticeable in a passing siren,
  as well as the redshift seen by astromers.
\item
  Arrange the following electro-magnetic radiation in order of increasing
  frequency: X-rays, gamma rays, infrared radiation, and visible light.
  \par\smallskip
  Correct order is: gamma rays, X-rays, visible light, and infrared radiation
\item
  State Snell's Law.
  \begin{figure}[H]
    \centering
    \includestandalone[height = 2.5in, mode = image]{NUPOCphysicswaves9}
    \caption{A ray refracted through a medium.}
  \end{figure}
  Let \(n_1\) be the first medium and \(n_2\) be the second medium.
  Let \(v_1 = \frac{c}{n_1}\) and \(v_2 = \frac{c}{n_2}\) where \(v_i\) is the
  speed of light in the ith medium.
  Then \(n_1\), \(n_2\geq 1\).
  The time taken to move from \(P\) to \(Q\) is
  \[
  T = \frac{d_1}{v_1} + \frac{d_2}{v_2}.
  \]
  By the distance formula, \(d_1 = \sqrt{A^2 + x^2}\) and
  \(d_2 = \sqrt{B^2 + (\ell - x)^2}\).
  Thus,
  \[
  T = \frac{\sqrt{A^2 + x^2}}{v_1} + \frac{\sqrt{B^2 + (\ell - x)^2}}{v_2}.
  \]
  Next, let's minimize the transient time; that is, we need to set
  \(\frac{dT}{dx} = 0\).
  \[
  \frac{dT}{dx} = \frac{x}{v_1\sqrt{A^2 + x^2}} -
  \frac{\ell - x}{v_2\sqrt{B^2 + (\ell - x)^2}} = 0
  \]
  Then \(\sin(\theta_1) = \frac{x}{\sqrt{A^2 + x^2}}\) and
  \(\sin(\theta_2) = \frac{\ell - x}{\sqrt{B^2 + (\ell - x)^2}}\).
  \begin{gather*}
    \frac{dT}{dx} = \frac{\sin(\theta_1)}{v_1} - \frac{\sin(\theta_2)}{v_2}
    = 0\\
    \Rightarrow\frac{\sin(\theta_1)}{v_1} = \frac{\sin(\theta_2)}{v_2}\\
    \Rightarrow\frac{\sin(\theta_1)}{\sin(\theta_2)} = \frac{n_2}{n_1}
    \eqnumtag\label{snell}
  \end{gather*}
  where \cref{snell} is Snell's law.
\item
  Draw a picture of a fish in water and show where you would throw a spear to
  hit it.
  Where does the fish appear?
  Why?
  How do \(n\) and \(C\) relate to refraction?
  \begin{figure}[H]
    \centering
    \includestandalone[height = 2in, mode = image]{NUPOCphysicswaves10}
    \caption{A fish under water.}
  \end{figure}
  Let line \(PO\) where \(O\) is the origin be the line of site.
  Then \(Q\) is where the fish actually is and the red dashed line is where the
  fish would appear to be.
  Therefore, we will want to throw the spear along the solid black line.
  This occurs because we are going from the medium of air into water.
  The speed of light through the medium is \(v = \frac{c}{n}\) where \(n\) is
  the index of refraction.
  Thus, \(n = \frac{c}{v}\) where \(c\) is the speed of light.
\item
  Draw a concave and convex lens.
  What effect would each have on paraxial rays?
  Why?
  \begin{figure}[H]
    \centering
    \subcaptionbox{Convex lens\label{convex}}{
      \includestandalone[height = 2in, mode = image]{NUPOCphysicswaves11}}
    \qquad
    \subcaptionbox{Concave lens\label{concave}}{
      \includestandalone[height = 2in, mode = image]{NUPOCphysicswaves11b}}
    \caption{Paraxial rays moving through convex and concave lenses.}
  \end{figure}
  With a convex lens, the paraxial rays would be refracted towards the focus
  on the opposing side, see \cref{convex}.
  For the concave lencs, the paraxial rays would refracted along the angle of
  incoming sides focus, see \cref{concave}.
  This occurs because of Fermat's principle or the principle of least time.
  Fermat's principle says that the path taken between two points by a ray of
  light is the path that can be traversed in the least time.
  Snell's law is the solution to Fermat's principle.
  Therefore, concave and convex lenses do what they do based on the geometry
  of their lens grind, in conjuction with Snell's law.
\item
  What does a diffraction grating do, and what is it used for?
  Are there circumstances under which light must be considered a particle?
  When?
  \par\smallskip
  Diffraction grating separates light of different wave lengths.
  Think of a prism.
  Diffraction gratings are used in monochromators, spectrometers, and lasers to
  name a few.
  Ligth should be considered a particle in the photoelectric effect.
  More can be read on this topic here
  \url{http://en.wikipedia.org/wiki/Photoelectric_effect}
\end{enumerate}

%%% Local Variables:
%%% mode: latex
%%% TeX-master: t
%%% End:
