\documentclass[tikz, convert = false]{standalone}

\usepackage[utf8]{inputenx}%  http://ctan.org/pkg/inputenx
% Euler for math | Palatino for rm | Helvetica for ss | Courier for tt
\renewcommand{\rmdefault}{ppl}% rm
\linespread{1.05}% Palatino needs more leading
\usepackage[scaled]{helvet}% ss //  http://ctan.org/pkg/helvet
\usepackage{courier}% tt // http://ctan.org/pkg/courier
\usepackage{eulervm}  %  http://ctan.org/pkg/eulervm
% a better implementation of the euler package (not in gwTeX)
\normalfont%
\usepackage[T1]{fontenc}%  http://ctan.org/pkg/fontenc
\usepackage{textcomp}%  http://ctan.org/pkg/textcomp

\usepackage{pgfplots}
\pgfplotsset{compat = 1.10}

\begin{document}
\begin{tikzpicture}
  \begin{axis}[
    axis x line = center,
    axis y line = left,
    xmin = -3,
    ymin = -7,
    xmax = 2,
    ymax = 0,
    legend style = {at = {(.5, .75)}, anchor = north}
    ]
    \addplot[blue, samples = 500, smooth] gnuplot[domain = -3:2] {x^2 + x - 6};
    \addplot[red, samples = 500, domain = -3:2] {-4};
    \legend{$f(x) = x^2 + x - 6$, $f(x) = -4$};
  \end{axis}
\end{tikzpicture} 
\end{document}
%%% Local Variables:
%%% mode: latex
%%% TeX-master: t
%%% End:
